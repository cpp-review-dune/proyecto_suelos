

\begin{frame}[fragile]
\frametitle{Numerische Lösung des Pendels}
\begin{itemize}
\item Volles Modell für das Pendel aus der Einführung:
  \begin{gather*}
    \frac{d^2\phi(t)}{d t^2} = - \frac{g}{l} \sin(\phi(t)) \qquad \forall t>0,\\
    \phi(0) = \phi_0, \qquad \frac{d \phi}{d t}(0) = u_0.
  \end{gather*}
\item Umschreiben in System erster Ordnung:
  \begin{align*}
    \frac{d\phi(t)}{d t} &= u(t), & \frac{d^2\phi(t)}{d t^2} &=
    \frac{d u(t)}{d t} = - \frac{g}{l} \sin(\phi(t)).
  \end{align*}
\item Eulerverfahren für $\phi^n = \phi(n\Delta t)$, $u^n = u(n\Delta t)$:
  \begin{align*}
    \phi^{n+1} &= \phi^n + \Delta t \, u^n & \phi^0 &= \phi_0\\
    u^{n+1} &= u^n -\Delta t \, (g/l) \, \sin(\phi^n) & u^0 &= u_0
  \end{align*}
\end{itemize}
\end{frame}

\begin{frame}[fragile]
\frametitle{Pendel (expliziter Euler)}
\lstinputlisting[basicstyle=\ttfamily\scriptsize,numbers=left,
numberstyle=\tiny, numbersep=5pt]{../examples/progkurs/pendelnumerisch.cc}
\end{frame}

\subsection{Funktionen}

\begin{frame}[fragile]
\frametitle{Funktionsaufruf und Funktionsdefinition}
\begin{itemize}
\item In der Mathematik gibt es das Konzept der \textsl{Funktion}.
\item In C++ auch.
\item Sei $f : \mathbb{R} \to \mathbb{R}$, z.B. $f(x) = x^2$.
\item Wir unterscheiden den \textsl{Funktionsaufruf}
{\scriptsize\begin{lstlisting}{}
double x,y;
y = f(x);
\end{lstlisting}}
\item und die \textsl{Funktionsdefinition}. Diese sieht so aus:

\medskip
\textsl{Ergebnistyp} \textsl{Funktionsname} \lstinline{(} \textsl{Argumente} \lstinline{)}

\lstinline!{! \textsl{Funktionsrumpf} \lstinline!}!

\medskip
\item Beispiel:
{\scriptsize\begin{lstlisting}{}
double f (double x)
{
  return x*x;
}
\end{lstlisting}}
\end{itemize}
\end{frame}

\begin{frame}[fragile]
\frametitle{Komplettbeispiel zur Funktion}
\lstinputlisting[basicstyle=\ttfamily\scriptsize,numbers=left,
numberstyle=\tiny, numbersep=5pt]{../examples/progkurs/funktion.cc}
\begin{itemize}
\item Funktionsdefinition muss vor Funktionsaufruf stehen.
\item Formales Argument in der Funktionsdefinition entspricht einer Variablendefinition.
\item Beim Funktionsaufruf wird das Argument (hier) \textsl{kopiert}.
\item \lstinline{main} ist auch nur eine Funktion.
\end{itemize}
\end{frame}

\begin{frame}[fragile]
\frametitle{Weiteres zum Verständnis der Funktion}
\begin{itemize}
\item Der Name des formalen Arguments in der Funktionsdefinition
  ändert nichts an der Semantik der Funktion (Sofern es überall
  geändert wird):

{\scriptsize\begin{lstlisting}{}
double f (double y)
{
  return y*y;
}
\end{lstlisting}}

\item Das Argument wird hier kopiert, d.h.:
{\scriptsize\begin{lstlisting}{}
double f (double y)
{
  y = 3*y*y;
  return y;
}

int main ()
{
  double x(3.0),y;
  y = f(x); // ändert nichts an x !
}
\end{lstlisting}}
\end{itemize}
\end{frame}

\begin{frame}[fragile]
\frametitle{Weiteres zum Verständnis der Funktion}
\begin{itemize}
\item Argumentliste kann leer sein (wie in der Funktion
  \lstinline{main}):
{\scriptsize\begin{lstlisting}{}
double pi ()
{
  return 3.14;
}

y = pi(); // Klammern sind erforderlich!
\end{lstlisting}}

\item Der Rückgabetyp \lstinline{void} bedeutet \glqq{}keine
  Rückgabe\grqq{}

{\scriptsize\begin{lstlisting}{}
void hello ()
{
  std::cout << "hello" << std::endl;
}

hello();
\end{lstlisting}}
\item Mehrere Argument werden durch Kommata getrennt:

{\scriptsize\begin{lstlisting}{}
double g (int i, double x)
{
  return i*x;
}
std::cout << g(2,3.14) << std::endl;
\end{lstlisting}}
\end{itemize}
\end{frame}

\begin{frame}[fragile]
\frametitle{Pendelsimulation als Funktion}
\lstinputlisting[basicstyle=\ttfamily\scriptsize,numbers=left,
numberstyle=\tiny, numbersep=5pt]{../examples/progkurs/pendelmitfunktion.cc}
\end{frame}

\begin{frame}[fragile]
\frametitle{Aufgabe 4}
\begin{enumerate}
\item Schreibt eine Funktion, die die Fakultät einer (natürlichen) Zahl berechnet und ausgibt.
\item Plottet die Funktion mit Gnuplot im Bereich [1,10]. 
\end{enumerate}
\end{frame}

\begin{frame}[fragile]
\frametitle{Funktionsschablonen}
\begin{itemize}
\item Oft macht eine Funktion mit Argumenten verschiedenen Typs einen Sinn.
\item \lstinline!double f (double x) {return x*x;}! macht auch mit
  \lstinline{float}, \lstinline{int} oder \lstinline{mpf_class} Sinn.
\item Man könnte die Funktion für jeden Typ definieren. Das ist
  natürlich sehr umständlich. (Es darf mehrere Funktionen gleichen
  Namens geben, sog. \textsl{overloading}).
\item In C++ gibt es mit Funktionsschablonen (engl.: \textsl{function
  templates}) eine Möglichkeit den Typ variabel zu lassen:
{\scriptsize\begin{lstlisting}{}
template<typename T>
T f (T y)
{
  return y*y;
}
\end{lstlisting}}
\item \lstinline{T} steht hier für einen beliebigen Typ.
\end{itemize}
\end{frame}

\begin{frame}[fragile,allowframebreaks]
\frametitle{Pendelsimulation mit Templates}
\lstinputlisting[basicstyle=\ttfamily\scriptsize,numbers=left,
numberstyle=\tiny, numbersep=5pt]{../examples/progkurs/pendelmitfunktionstemplate.cc}
\end{frame}


\begin{frame}[fragile]
\frametitle{Headerdateien}
\begin{itemize}
\item Das tolle an Funktionen ist, dass man sie wiederverwenden kann.
\item Es gibt ganze Sammlungen an Funktionen, sogenannte Funktionsbibliotheken
\item Werden in separaten Headerdateien gespeichert. (Dateiendung .h,.hh, o."a.)
\item Um Funktionen (und anderes) aus einer Headerdatei zu verwenden:
\item \#include "Dateiname"
\item Es gibt eine Standardbibliothek von Headerdateien (z.b. cmath,string,iostream)

\end{itemize}
\end{frame}


\begin{frame}[fragile]
\frametitle{Headerdateien}
{\scriptsize\begin{lstlisting}{}
#include <iostream> // Standardbibliothek (Name in <>)
#include <cmath>
#include "hdnum.hh" // Eigene Headerdatei (Dateiname in " \ ")

int main()
{
	float zahl = 5.0;
	zahl = sqrt(zahl); //sqrt aus cmath 
	std::cout << zahl << std::endl; 
	// std::cout, std::endl aus iostream
	hdnum::vector(2,1.0); // Aus hdnum.hh
}
\end{lstlisting}}
\end{frame}

\begin{frame}[fragile]
\frametitle{Headerdateien}
\begin{itemize}
\item Nur \#include in die Datei zu schreiben reicht nicht!
\item Beim kompilieren muss noch der Ort der Headerdatei angegeben werden
\item bei dem kompilierbefehl -I<Verzeichnis> einfügen
\item z.b. g++ -o hallohdnum -I../../ hallohdnum.cc
\item ../../ bedeutet Oberverzeichnis des Oberverzeichnisses (2x ../)

\end{itemize}
\end{frame}


\begin{frame}[fragile]
\frametitle{Aufgabe 5}
\begin{enumerate}
\item Erstellt eine Headerdatei für eure Fakultätsfunktion aus Aufgabe 3. (z.b. fakultaet.hh)
\item Schreibt eure Funktionsdefinition in die Headerdatei, und kommentiert sie in der Hauptdatei aus.
\item Bindet eure headerdatei mit \#include ein und führt eure Funktion in der Datei aus.
\end{enumerate}
\end{frame}