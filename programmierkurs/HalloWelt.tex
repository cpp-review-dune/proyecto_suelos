
\begin{frame}[fragile]
\frametitle{Was ist Programmierung?}
\begin{itemize}
\item Ein Programm ist im praktischen Sinne eine Aneinanderreihung von Befehlen,
die ein Computer (in Reihenfolge) auszuführen hat.
\item Es gibt verschiedene Programmiersprachen in denen die Befehle geschrieben werden können.
\item Wir verwenden C++.
\item In der Regel (auch in C++) ist Gro\ss - und Kleinschreibung zu beachten.
\end{itemize}
\end{frame}

\begin{frame}[fragile]
\frametitle{Workflow}
C++ ist eine \glqq{}kompilierte\grqq{} Sprache. Um ein Programm zur
Ausführung zu bringen sind folgende Schritte notwendig:
\begin{enumerate}
\item Erstelle/Ändere den Programmtext mit einem \textbf{Editor}.
\item Übersetze den Programmtext mit dem \textbf{C++-Übersetzer}
  (auch C++-Compiler) in ein Maschinenprogramm.
\item Führe das Programm aus. Das Programm gibt sein Ergebnis auf dem
  Bildschirm oder in eine Datei aus.
\item Falls Ergebnis nicht korrekt, gehe nach 1!
\end{enumerate}
\end{frame}


\begin{frame}[fragile]
\frametitle{Hallo Welt !}
Öffne die Datei \lstinline{hallohdnum.cc} mit einem Editor:
\\ \lstinline{$ gedit hallohdnum.cc} %$
\lstinputlisting[basicstyle=\ttfamily\scriptsize,numbers=left,
numberstyle=\tiny, numbersep=5pt]{../examples/progkurs/hallohdnum.cc}
\begin{itemize}
\item \lstinline{iostream} ist eine sog. \glqq{}Headerdatei\grqq{}
\item \lstinline!#include! erweitert die \glqq{}Basissprache\grqq{}.
\item \lstinline!int main ()! braucht man immer: \glqq{}Hier geht's los\grqq{}.
\item \lstinline!{ ... }! klammert Folge von Anweisungen.
\item Anweisungen werden durch Semikolon abgeschlossen.
\end{itemize}
\end{frame}


\begin{frame}[fragile]
\frametitle{Hallo Welt laufen lassen}
\begin{itemize}
\item Gebe folgende Befehle ein:
\begin{itemize}
\item Zum Kompilieren des Programms:
{\footnotesize\begin{lstlisting}{}
$ g++ -o hallohdnum -I../../  hallohdnum.cc
\end{lstlisting}}
\item g++ ist unser C++-Compiler
\item -o hallohdnum spezifiziert die dabei entstehende Datei
\item -I../../ spezifiziert das Verzeichnis der Headerdateien
\item hallohdnum.cc ist die Datei, die wir kompilieren wollen
\item Zum Ausführen des Programms:
{\small\begin{lstlisting}{}
$ ./hallohdnum
\end{lstlisting}}
\end{itemize}
\item Dies sollte dann die folgende Ausgabe liefern:
{\small\begin{lstlisting}{}
Numerik 0 ist ganz leicht!
1+1=2
\end{lstlisting}}
\end{itemize}
\end{frame}

\note{Demonstration: Hallo Welt ausführen einmal vormachen.}

\subsection{Variablen und Typen}

\begin{frame}[fragile]
\frametitle{(Zahl-) Variablen}
\begin{itemize}
\item Aus der Mathematik: \glqq{}$x\in M$\grqq{}. Variable $x$ nimmt einen
  beliebigen Wert aus der Menge $M$ an.
\item Geht in C++ mit: \lstinline{    M x;}
\item \textbf{Variablendefinition}:
\lstinline{x} ist eine Variable vom \textbf{Typ} \lstinline{M}.
\item Mit \textbf{Initialisierung}: \lstinline{    M x(0);}
\item Wert von Variablen der \glqq{}eingebauten\grqq{} Typen ist sonst
  nicht definiert.
\end{itemize}
\lstinputlisting[basicstyle=\ttfamily\scriptsize,numbers=left,
numberstyle=\tiny, numbersep=5pt]{../examples/progkurs/zahlen.cc}
\end{frame}


\begin{frame}[fragile]
\frametitle{Datentypen}
Datentypen sind sozusagen Wertebereiche der Variablen, bestimmen also welche Werte eine Variable annehmen kann. \\
Die wichtigsten Datentypen:
\begin{itemize}
\item int: Ganze Zahl im Wertebereich [$-2^{31}$,$2^{31}-1$]
\item unsigned int: Natürliche Zahl im Wertebereich [0,$2^{32}-1$]
\item float: Gleitkommazahl mit einfacher Präzision \\ im Bereich [-3.4e38, 3.4e38]
\item double: float mit doppelter Präzision\\ im Bereich [-1.80e+308, 1.80e+308]
\item char: (einzelne) ASCII-Zeichen ("Buchstaben"), \\ z.b. "b","f","C","\%",
\item string: Zeichenketten, z.b. "Hallo","fjofj"
\end{itemize}
\end{frame}

\begin{frame}[fragile]
\frametitle{Zuweisung}
\begin{itemize}
\item Den Wert von Variablen kann man ändern. Sonst wäre es langweilig
  :-)
\item Dies geht mittels Zuweisung:
{\scriptsize\begin{lstlisting}{}
double x(3.14); // Variablendefinition mit Initialisierung
double y;       // uninitialisierte Variable
y = x;          // Weise y den Wert von x zu
x = 2.71;       // Weise x den Wert 2.71, y unverändert
y = (y*3)+4;    // Werte Ausdruck rechts von = aus
                // und weise das Resultat y zu!
\end{lstlisting}}
\end{itemize}
\end{frame}

\begin{frame}[fragile]
\frametitle{Blöcke}
\begin{itemize}
\item Block: Sequenz von Variablendefinitionen und Zuweisungen in
  geschweiften Klammern.
{\scriptsize\begin{lstlisting}{}
{
  double x(3.14);
  double y;
  y = x;
}
\end{lstlisting}}
\item Blöcke können rekursiv geschachtelt werden.
\item Eine Variable ist nur in dem Block \textsl{sichtbar} in dem sie
  definiert ist sowie in allen darin enthaltenen Blöcken:
{\scriptsize\begin{lstlisting}{}
{
  double x(3.14);
  {
    double y;
    y = x;
  }
  y = (y*3)+4; // geht nicht, y nicht mehr sichtbar.
}
\end{lstlisting}}
\end{itemize}
\end{frame}

\begin{frame}[fragile]
\frametitle{Whitespace}
\begin{itemize}
\item Das Einrücken von Zeilen dient der besseren Lesbarkeit,
  notwendig ist es (fast) nicht.
\item \lstinline{#include}-Direktiven müssen \textsl{immer} einzeln
  auf einer Zeile stehen.
\item Ist das folgende Programm lesbar?
\lstinputlisting[basicstyle=\ttfamily\scriptsize,numbers=left,
numberstyle=\tiny, numbersep=5pt]{../examples/progkurs/whitespace.cc}
\end{itemize}
\end{frame}

\begin{frame}[fragile]
\frametitle{Arithmetik}
Mit Variablen lassen sich sogenannte Operationen durchführen, z.b. auch Arithmetik:
\lstinputlisting[basicstyle=\ttfamily\scriptsize,numbers=left,
numberstyle=\tiny, numbersep=5pt]{../examples/progkurs/operationen.cc}
\end{frame}

\begin{frame}[fragile]
\frametitle{Arrays}
Eine Variable kann auch mehrere Werte speichern,
wenn man es als Array definiert:
\begin{itemize}
\item Ein Array ist eine Liste von Variablen.
\item Wird definiert durch Typ Variable[Grö\ss e];
\item z.b. int a[10];
\item Der Zugriff erfolgt über Variable[index];
\item Der erste index ist 0, daher ist der letzte index Grö\ss e-1!
\item Beispiel: a[0], a[4], a[8], a[9]
\item a[10] ist au\ss erhalb des Arrays und produziert einen Fehler!
\end{itemize}
\end{frame}

\begin{frame}[fragile]
\frametitle{Beispiel: Fibonacci-Zahlen}
{\scriptsize\begin{lstlisting}{}

int main()
{
int x[10]; // Definition eines double Arrays mit 10 Werten

x[0] = 1; // Initialisierung!
x[1] = 1;
x[2] = x[1] + x[0];
x[3] = x[2] + x[1];

// usw....

}
\end{lstlisting}}
\end{frame}

\begin{frame}[fragile]
\frametitle{Ein- und Ausgabe}
Manchmal möchte man bei der Ausführung wissen welche Werte Variablen haben,
oder auch Variablen durch Eingabe festlegen.
\\
\begin{itemize}
\item C++ benutzt zur Ein- und Ausgabe sogenannte streams
\item Findet man im Headerfile " \ iostream "
\item std::cout << Variable << std::endl;  zur Ausgabe der Variable
\item std::cin >> Variable; zur Eingabe in die Variable (auf Datentyp achten!)
\item Das std::endl dient dem Zeilenumbruch
\item Eigentlich optional aber oft empfehlenswert
\end{itemize}
\end{frame}


\begin{frame}[fragile]
\frametitle{Aufgabe 2}
\begin{enumerate}
\item Öffnet operationen.cc im Verzeichnis examples/progkurs mit einem Editor. (z.b. mit Befehl vi operationen.cc)
\item Gebt nach jeder Zuweisung an c den Wert von c im Format "c = (Wert von c)" \ aus.
\item Kompiliert operationen.cc so wie oben hallohdnum.cc.
\item Überlegt euch ob die Werte euren Erwartungen entsprechen.
\end{enumerate}
\end{frame}

\note{Demonstration: Benutzung von vim einmal vormachen.}