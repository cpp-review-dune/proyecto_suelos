\nocite{*}
\mode<presentation>
{
  \begin{frame}
    \titlepage
  \end{frame}
}
\mode<article>
{
\maketitle
}

%\mode<presentation>{
%\begin{frame}<presentation>
%\frametitle{Outline}
%\tableofcontents[section,sectionstyle=show/show,subsectionstyle=hide/hide/hide]
%\end{frame}
%}


%%%%%%%%%%%%%%%%%%%%%%%%%%%%%%%%%%%%%%%%%%%%%%%%%%%%%%%%%%%%%%%
% Die Kapitel
%%%%%%%%%%%%%%%%%%%%%%%%%%%%%%%%%%%%%%%%%%%%%%%%%%%%%%%%%%%%%%%



\begin{frame}[fragile]
\frametitle{Beschreibung des Kurses}
\begin{itemize}
\item Kurze Einführung in LINUX
\item Einrichten eines LINUX-Subsystems für Windows 10
\item Programmierung in C++ unter LINUX
\item Idee des Kurses: \glqq{}Lernen an Beispielen\grqq{}, keine
  rigorose Darstellung
\item Keine Vorkenntnisse erforderlich
\item Übungsaufgaben zur Vertiefung
\item Möglichkeit Fragen zu stellen
\end{itemize}
\end{frame}

\begin{frame}[fragile]
\frametitle{Fragen über CodiMD}
\begin{enumerate}
\small
\item \url{https://codimd.mathphys.stura.uni-heidelberg.de/}
\item Notiz erstellen und mir dann die url schicken.
\item Beispiel für das Format der Nachricht:
\item \url{https://codimd.mathphys.stura.uni-heidelberg.de/ProgrammierkursTest?view}
\item Am besten Programmcode + Fehlermeldung (falls vorhanden) + Problembeschreibung (Wo euer Verständnisproblem liegt)

\end{enumerate}
\end{frame}

\note{Hier einmal CodiMD zeigen.}

\begin{frame}[fragile]
\frametitle{Inhaltsverzeichnis}
\begin{itemize}
\footnotesize
\item Kurze Einführung in LINUX
\item Ein erstes Programm
\item Variablen
\item Ein- und Ausgabe
\item If-Statements
\item for- und while-Schleifen
\item Funktionen
\item Headerdateien
\item Klassen
\item HDNum
\item Hilfe zur Selbsthilfe
\end{itemize}
\end{frame}


\begin{frame}[fragile]
\frametitle{Warum Programmieren lernen?}
\begin{itemize}
\item Auch interessant für das Lehramt:
\item Hilft beim (detaillierten) Verständnis von Algorithmen
\item Zur Kontrolle von Lösungen
\item Visualisierung
\item Für die Lehre
\end{itemize}
\end{frame}

\begin{frame}[fragile]
\frametitle{Beispiel: Lösung eines LGS}
\begin{itemize}
\item Für kleine Matrizen noch von Hand machbar
\item Für N = 100+ oder sogar 1000+ relativ schwierig
\item Au\ss erdem sehr Fehleranfällig
\item Ein Programm schafft auch sehr gro\ss e Matrizen in sehr kurzer Zeit
\item Relativ fehlerfrei, au\ss er natürlich Numerik (Rundungsfehler, etc.)
\end{itemize}
\end{frame}

%\note{Demonstration: LGS mittels Programmierung}

\begin{frame}[fragile]
\frametitle{Linux-Basics}
Man hat zwei Möglichkeiten Linux zu bedienen:
\begin{itemize}
\item graphische Benutzeroberfläche (GUI)
\item Texteingabe (Shell)
\item GUI einfach zu erlernen, ähnlich wie bei Windows
\item Shell ist deutlich produktiver, aber erfordert Lernen der Syntax (am Anfang etwas ungewohnt)
\end{itemize}
\end{frame}

\note{Einmal Terminal vormachen, Arbeitsverzeichnis, Befehle eingeben}

\begin{frame}[fragile]
\frametitle{Wichtige Befehle}
cd - Wechselt das momentane Arbeitsverzeichnis.
\begin{itemize}
\item Syntax: cd <Dateipfad>
\item Pfadangabe meist relativ zum Arbeitsverzeichnis
\item Relative Pfadangabe in Linux:
\item Momentanes Verzeichnis: .
\item Unterverzeichnis: ./<Verzeichnis>
\item Oberverzeichnis: ..
\end{itemize}
\end{frame}

\note{cd-Befehl und relative Pfadangabe zeigen}

\begin{frame}[fragile]
\frametitle{Wichtige Befehle}
\begin{itemize}
\item ls - Zeige Inhalt des aktuellen Verzeichnisses.
\item mkdir <Verzeichnis> - Erstelle neues Verzeichnis
\item cp <datei1> <datei2> - Kopiere datei1 auf datei2 (datei2 kann verzeichnis sein)
\item mv wie cp nur verschieben anstatt kopieren
\item rm <datei> - datei löschen
\item rm -rf Lösche Verzeichnis mit Inhalt
\end{itemize}
\end{frame}

\note{Befehle vormachen}

\begin{frame}
\frametitle{Weitere Befehle}
\begin{itemize}
\item sudo <Befehl> Befehl als Administrator ausführen
\item man <Befehl> - zeigt einem die Bedienungsanweisung für den Befehl
\item Für alles Weitere: linux cheat sheet bei google eingeben
\item Zusammenstellung der wichtigsten Befehle
\end{itemize}
\end{frame}

\note{Befehle vormachen, sudo, man}

\begin{frame}[fragile]
\frametitle{LINUX unter Windows 10}
\begin{itemize}
\item Nicht alle haben einen LINUX-Rechner zuhause.
\item Windows 10 hat dafür einen "LINUX-Emulator"
\item Installation:
\begin{enumerate} 
\item Windows Powershell öffnen
\item Enable-WindowsOptionalFeature -Online -FeatureName Microsoft-Windows-Subsystem-Linux eingeben
\item Computer Neustarten falls gefordert
\item Im Microsoft Store Ubuntu runterladen \& installieren
\item Auch zu finden unter: https://docs.microsoft.com/de-de/windows/wsl/install-win10
\end{enumerate}
\end{itemize}
\end{frame}

\note{ Einrichten der Arbeitsumgebung, git}

\begin{frame}[fragile]
\frametitle{Aufgabe 1: Einrichten eurer Arbeitsumgebung}
\begin{itemize}
\item Entweder Ubuntu oder Ubuntu unter Windows 10 installieren
\item Git installieren
\item HDNum repository runterladen
\item Befehl: git clone \url{https://parcomp-git.iwr.uni-heidelberg.de/Teaching/hdnum.git}
\item Eigenen Ordner erstellen
\end{itemize}
\end{frame}