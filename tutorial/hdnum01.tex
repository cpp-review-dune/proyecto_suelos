\documentclass[a4paper,11pt]{article}
\usepackage[spanish]{babel}
\usepackage[utf8]{inputenc}
\usepackage[T1]{fontenc}
\usepackage[a4paper,total={150mm,240mm}]{geometry}
\usepackage{amsmath}
\usepackage{amsfonts}
\usepackage{amsthm}
\usepackage{amscd}

\usepackage[envcountsect,noamsthm]{beamerarticle}

\usepackage{grffile}
\usepackage{tikz}
\usetikzlibrary{3d,calc}
\usepackage{eurosym}
\usepackage{graphicx}
\usepackage{color}
\definecolor{listingbg}{gray}{0.95}
\usepackage{listings}
\lstset{language=C++, basicstyle=\ttfamily,
  keywordstyle=\color{black}\bfseries, tabsize=4, stringstyle=\ttfamily,
  commentstyle=\itshape, extendedchars=false,backgroundcolor=\color{listingbg},escapeinside={/*@}{@*/}}
\usepackage{paralist}
\usepackage{curves}
\usepackage{calc}
\usepackage{picinpar}
\usepackage{enumerate}
\usepackage{algpseudocode}
\usepackage{bm}
\usepackage{multibib}
\usepackage{hyperref}
\usepackage{textcase}
\usepackage{nicefrac}
\usepackage[titletoc,toc,title]{appendix}

\theoremstyle{definition}
\newtheorem{exm}{Example}
\newtheorem{thm}{Theorem}
\newtheorem{cor}{Corollary}
\newtheorem{obs}{Observation}

\title{HDNUM\\ CPP Review Proyecto para Latinoamérica}
\author{Peter Bastian\\
  Universität Heidelberg, \\
  Traducido por John Jairo Leal G,\\
  Universidad Nacional de Colombia, \\
  Línea de investigación en Modelamiento Matemático\\
}
\date{\today}

\begin{document}

\maketitle



\begin{abstract}
Este manual de Hdnum, está basado en la propuesta presentada por la universidad de 
Heidelberg Alemania en el curso de Lima, Perú en 2020, para introducirse en el manejo de las soluciones de ecuaciones
diferenciales ordinarias y parciales.
\end{abstract}

\tableofcontents{tocdepth}{5}

\section{Introducción}
Haremos un breve resumen de HDNUM
\subsection{Que es HDNUM?}
La biblioteca numérica de Heildeberg HDNUM, es una biblioteca basada en C++ para realizar ejercicios prácticos en clases
magistrales, que incluye métodos numéricos para resolver ecuaciones diferenciales ordinarias, la actual versión está 
disponible en :
\begin{center}
\url{https://parcomp-git.iwr.uni-heidelberg.de/Teaching/hdnum}
\end{center}

Que tiene un control de vesiones administrada por \lstinline{git} .
Spezifische Versionen können auf der jeweiligen Vorlesungswebseite veröffentlicht werden.

Los objetivos en el desarrollo de HDNUM fueron i) la facilidad de uso (incluida la instalación simple), ii) la demostración de la programación orientada a objetos en la solución de métodos numéricos, así como la posibilidad de realizar cálculos con cualquier grado de precisión sobre la base de Gnu Multiple Precision Biblioteca. HDNUM ofrece actualmente las siguientes funciones:
\begin{enumerate}[1)]
\item Clases para vectores y matrices
\item Solución de sistemas de ecuaciones lineales
\item Solución de sistemas de ecuaciones no lineales
\item Solución de la ecuación de Poisson utilizando diferencias finitas
\end{enumerate}

\subsection{Instalación}

HDNUM es una biblioteca de ''solo encabezado'' y no requiere ninguna instalación más que descargar los archivos. La versión actual se puede encontrar usando el siguiente comando para descargarla:

\begin{lstlisting}[basicstyle=\ttfamily\footnotesize,language=bash,frame=single]
$ git clone https://parcomp-git.iwr.uni-heidelberg.de/Teaching/hdnum.git
\end{lstlisting}

Para realizar esto se requiere el programa \lstinline{git}, el cual está disponible gratuitamente para todos los sistemas operativos. Alternativamente también hay un archivo comprimido que se puede descargar \lstinline{tar} en la página del evento.
\begin{lstlisting}[basicstyle=\ttfamily\small,frame=single]
$ tar zxvf hdnum-XX.tgz
\end{lstlisting}
Los siguientes archivos y subdirectorios se pueden encontrar en el directorio instalado o descomprimido:
\begin{itemize}
\item \lstinline{hdnum.hh}: Este archivo encabezado debe estar integrado en los programa \lstinline{C++} para poder utilizar HDNUM.
\item 
El directorio \lstinline{mystuff} está destinado a sus programas, pero, por supuesto, puede utilizar cualquier otro directorio. Lo único importante es que el compilador tenga el archivo \lstinline{hdnum.hh} que encontró. En el registro mystuff ya es un programa de muestra para comenzar de inmediato. Este programa se puede compilar de la siguiente forma:

\begin{lstlisting}[basicstyle=\ttfamily\small,frame=single]
$ cd mystuff
$ g++ -I.. -o example example.cc
\end{lstlisting}
Otra forma es utilizar el comando \lstinline{make}, el cual ejecutará el archivo \lstinline{Makefile} que está en la carpeta, y el cual compilará el programa ejemplo, generando el archivo ejecutable \lstinline{example} que lo puede correr con \lstinline{./example}.

Estos comandos requieren que el compilador GNU C ++ esté instalado en su sistema. En Windows o para otros compiladores, debe adaptar los comandos en consecuencia.

\item El directorio \lstinline{examples} en la carpeta HDNUM contiene muchos ejemplos ordenados para el curso de programación, \lstinline{num0} y \lstinline{num1}.
\item El directorio \lstinline{src} en la carpeta HDNUM contiene el código fuente de la biblioteca HDNUM. Estos archivos son utilizados por hdnum.hh cuando sea invocado.
\item El directorio \lstinline{programmierkurs} en la carpeta HDNUM contiene el código fuente de este documento
\item El directorio \lstinline{tutorial}  en la carpeta HDNUM contiene las diapositivas del curso de programación.
\end{itemize}

\subsubsection*{GNU Biblioteca de Multiple Precisión}

HDNUM puede realizar cálculos con gran precisión. Esto requiere la biblioteca GNU Multiple Precision Library (GMP), que puede obtener de forma gratuita para muchos sistemas. Para poder utilizar GMP debe poner en el archivo \lstinline{hdnum.hh} la línea:
\begin{lstlisting}[basicstyle=\ttfamily\small,frame=single]
#define HDNUM_HAS_GMP 1
\end{lstlisting}
Además, las opciones del compilador pueden ser necesarias para que el compilador pueda encontrar las bibliotecas y los archivos de encabezado GMP. Entonces puede verse así:
\begin{lstlisting}[basicstyle=\ttfamily\footnotesize,frame=single]
$ g++ -I.. -I/opt/local/include -o example example.cc -L/opt/local/lib -lgmpxx -lgmp
\end{lstlisting}
% $

\section{Algebra Lineal}

\subsection{Vectores}

\begin{frame}[fragile]
\frametitle{\lstinline{hdnum::Vector<T>}}
\begin{itemize}
\item \lstinline{hdnum::Vector<T>} es una plantilla de Clase.
\item Convierte cualquier (número)  tipo de datos \lstinline{T} en un vector. 
\item También son posibles números complejos y muy precisos.
\item Los vectores se comportan como los conocemos por las matemáticas:
\begin{itemize}
\item Tiene $n$ componentes.
\item Inicia en el elemento $0$ hasta el elemento $n-1$ numerados consecutivamente.
\item Adición y multiplicación por un escalar.
\item Producto escalar y norma Euclidiana.
\item Multiplicación vector-matriz.
\end{itemize}
\item Los siguientes ejemplos se pueden encontrar en el archivo \lstinline{vektoren.cc}
\end{itemize}
\end{frame}

\begin{frame}[fragile]
\frametitle{Construcción y Acceso}
\begin{itemize}
\item Construcción con y sin Inicialización\\
{\footnotesize{\begin{lstlisting}{}
hdnum::Vector<float> x(10);        // Vector con 10 elementos
hdnum::Vector<double> y(10,3.14);  // Vector con 10 elementos inicializado
hdnum::Vector<float> a;            // Vector sin elementos
\end{lstlisting}}}
\item Vectores más específicos\\
{\footnotesize{\begin{lstlisting}{}
hdnum::Vector<std::complex<double> >
  cx(7,std::complex<double>(1.0,3.0));
mpf_set_default_prec(1024); // Establece precision para  mpf_class
hdnum::Vector<mpf_class> mx(7,mpf_class("4.44"));
\end{lstlisting}}}
\item Acceso a un elemento\\
{\footnotesize{\begin{lstlisting}{}
for (std::size_t i=0; i<x.size(); i=i+1)
  x[i] = i;                 // Acceso a cada elemento
\end{lstlisting}}}
\item El objeto vectorial se elimina al final del ciclo for.
\end{itemize}
\end{frame}

\begin{frame}[fragile]
\frametitle{Copia y Asignación}
\begin{itemize}
\item Constructor copia: crea una copia de otro vector
{\footnotesize{\begin{lstlisting}{}
hdnum::Vector<float> z(x); // z es copia de x
\end{lstlisting}}}
\item Asignación, ¡El tamaño también cambia!
{\footnotesize{\begin{lstlisting}{}
b = z;              // b copia los datos de z 
a = 5.4;            // asignacion a todos los elementos
hdnum::Vector<double> w;   // Vector sin elementos
w.resize(x.size()); // Redimiensiona el vector 
w = x;              // Copia los elementos
\end{lstlisting}}}
\item Extracto de vectores\\
{\footnotesize{\begin{lstlisting}{}
hdnum::Vector<float> w(x.sub(7,3));// w es una copia de x[7],...,x[9]
z = x.sub(3,4);             // z es una copia de x[3],...,x[6]
\end{lstlisting}}}
\end{itemize}
\end{frame}

\begin{frame}[fragile]
\frametitle{Cálculos y operaciones}
\begin{itemize}
\item Operaciones de espacio vectorial y producto escalar \\
{\footnotesize{\begin{lstlisting}{}
w += z;            // w = w+z
w -= z;            // w = w-z
w *= 1.23;         // Multiplicacion por escalar
w /= 1.23;         // Division por escalar
w.update(1.23,z);  // w = w + a*z
float s;
s = w*z;           // Producto escalar
\end{lstlisting}}}
\item Mostrando las salidas\\
{\footnotesize{\begin{lstlisting}{}
std::cout << w << std::endl;// Salida por pantalla
w.iwidth(2);                // Digitos en la salida del indice
w.width(20);                // Definiendo la cantidad de digitos
w.precision(16);            // Numero de posiciones decimales
std::cout << w << std::endl;// Imprimiendo w
std::cout <<cx << std::endl;// Imprimiendo complejos
std::cout <<mx << std::endl;// Funciona para mpf_class
\end{lstlisting}}}
\end{itemize}
\end{frame}

\begin{frame}[fragile]
\frametitle{Visualización}
{\footnotesize{\begin{lstlisting}{}
[   0]    1.204200e+01
[   1]    1.204200e+01
[   2]    1.204200e+01
[   3]    1.204200e+01

[ 0] 1.2042000770568848e+01
[ 1] 1.2042000770568848e+01
[ 2] 1.2042000770568848e+01
[ 3] 1.2042000770568848e+01
\end{lstlisting}}}
\end{frame}

\begin{frame}[fragile]
\frametitle{Funciones auxiliares}
{\footnotesize{\begin{lstlisting}{}
float d = norm(w);          // Norma Euclidea
d = w.two_norm();           // La misma norma
zero(w);                    // Igual que w=0.0
fill(w,(float)1.0);         // Igual que w=1.0
fill(w,(float)0.0,(float)0.1); // w[0]=0, w[1]=0.1, w[2]=0.2, ...
unitvector(w,2);            // Vector unitario cartesiano
gnuplot("test.dat",w);      // Salida a gnuplo: i w[i]
gnuplot("test2.dat",w,z);   // Saida a gnuplot: w[i] z[i]
\end{lstlisting}}}
\end{frame}

\begin{frame}[fragile]
\frametitle{Funciones}
\begin{itemize}
\item Ejemplo: Suma de todas las componentes \\
{\footnotesize{\begin{lstlisting}{}
double sum (hdnum::Vector<double> x) {
  double s(0.0);
  for (std::size_t i=0; i<x.size(); i=i+1)
    s = s + x[i];
  return s;
}
\end{lstlisting}}}
\item Versión mejorada de la función con \textbf{template} y uso por referencia\\
{\footnotesize{\begin{lstlisting}{}
template<class T>
T sum (const hdnum::Vector<T>& x) {
  T s(0.0);
  for (std::size_t i=0; i<x.size(); i=i+1)
    s = s + x[i];
  return s;
}
\end{lstlisting}}}
\item El uso por referencia es mejor para objetos grandes.
\end{itemize}
\end{frame}

\subsection{Matrices}

\begin{frame}[fragile]
\frametitle{\lstinline{hdnum::DenseMatrix<T>}}
\begin{itemize}
\item \lstinline{hdnum::DenseMatrix<T>} Es una plantilla o Template de clase.
\item Convierte un elemento tipo \lstinline{T} en una matriz.
\item También es posible incluir complejos y su precisión.
\item Las matrices son como las conocemos en matemáticas:
\begin{itemize}
\item Dimensión de $m\times n$ .
\item Inicia en elemento $0$ hasta $m-1$ o, hasta $n-1$ numerados consecutivamente.
\item El conjunto de las matrices $m\times n$ constituyen un espacio vectorial.
\item Multiplicación de vectores y de matrices.
\end{itemize}
\item Los siguientes ejemplos se pueden encontrar en el archivo \lstinline{matrizen.cc}
\end{itemize}
\end{frame}

\begin{frame}[fragile]
\frametitle{Construcción y acceso}
\begin{itemize}
\item Construcción con y sin inicialización \\
{\footnotesize{\begin{lstlisting}{}
hdnum::DenseMatrix<float> B(10,10);     // 10x10 Matrix sin inicializar
hdnum::DenseMatrix<float> C(10,10,0.0); // 10x10 Matrix inicializada
\end{lstlisting}}}
\item Acceso a elementos\\
{\footnotesize{\begin{lstlisting}{}
for (int i=0; i<B.rowsize(); ++i)
  for (int j=0; j<B.colsize(); ++j)
    B[i][j] = 0.0;          // Ahora la matriz B está inicializada
\end{lstlisting}}}
\item El objeto matriz se elimina al final del ciclo.
\end{itemize}
\end{frame}

\begin{frame}[fragile]
\frametitle{Copia y asignación}
\begin{itemize}
\item Constructor copia: Crea una matriz como copia de otra
{\footnotesize{\begin{lstlisting}{}
hdnum::DenseMatrix<float> D(B); // D copia de B
\end{lstlisting}}}
\item Asignación y tamaño de las copias
{\footnotesize{\begin{lstlisting}{}
hdnum::DenseMatrix<float> A(B.rowsize(),B.colsize()); //Tamano correcto
A = B;                    // Copia
\end{lstlisting}}}
\item Extractos de matrices (Submatrices)\\
{\footnotesize{\begin{lstlisting}{}
hdnum::DenseMatrix<float> F(A.sub(1,2,3,4));// 3x4  Submatriz (1,2)
\end{lstlisting}}}
\end{itemize}
\end{frame}

\begin{frame}[fragile]
\frametitle{Calculando con matrices}
\begin{itemize}
\item Operaciones de espacio vectorial: Ojo, las matrices deben
tener el tamaño correcto!)\\
{\footnotesize{\begin{lstlisting}{}
A += B;           // A = A+B
A -= B;           // A = A-B
A *= 1.23;        // Multiplicación por escalar
A /= 1.23;        // División por escalar
A.update(1.23,B); // A = A + s*B
\end{lstlisting}}}
\item Matrices, Vectores y multiplicación de matrices\\
{\footnotesize{\begin{lstlisting}{}
hdnum::Vector<float> x(10,1.0); // Construimos dos vectores
hdnum::Vector<float> y(10,2.0);
A.mv(y,x);               // y = A*x
A.umv(y,x);              // y = y + A*x
A.umv(y,(float)-1.0,x);  // y = y + s*A*x
C.mm(A,B);               // C = A*B
C.umm(A,B);              // C = C + A*B
A.sc(x,1);               // Hace x la primera columna de A
A.sr(x,1);               // Hace x la primera fila de A
float d=A.norm_infty();  // Halla la norma de la suma de las filas
d=A.norm_1();            // Halla la norma de la suma de las columnas
\end{lstlisting}}}
\end{itemize}
\end{frame}

\begin{frame}[fragile]
\frametitle{Funciones de salida auxiliares}
\begin{itemize}
\item Salida de matrices \\
{\footnotesize{\begin{lstlisting}{}
std::cout << A.sub(0,0,3,3) << std::endl;// Impresion
A.iwidth(2);                // Digitos en la salida del indice
A.width(10);                // Numero total de digitos
A.precision(4);             // Posiciones decimales
std::cout << A << std::endl;// Nueva publicacion
\end{lstlisting}}}
\item Algunas funciones auxiliares 
{\footnotesize{\begin{lstlisting}{}
identity(A);
spd(A);
fill(x,(float)1,(float)1);
vandermonde(A,x);
\end{lstlisting}}}
\end{itemize}
\end{frame}

\begin{frame}[fragile]
\frametitle{Salida de muestra}
{\footnotesize{\begin{lstlisting}{}
               0           1           2           3
  0   4.0000e+00 -1.0000e+00 -2.5000e-01 -1.1111e-01
  1  -1.0000e+00  4.0000e+00 -1.0000e+00 -2.5000e-01
  2  -2.5000e-01 -1.0000e+00  4.0000e+00 -1.0000e+00
  3  -1.1111e-01 -2.5000e-01 -1.0000e+00  4.0000e+00
\end{lstlisting}}}
\end{frame}

\begin{frame}[fragile]
\frametitle{Funciones con argumentos matriciales}
Ejemplo de una función de una matriz $A$ y de un vector  $b$
inicializados.

{\footnotesize{\begin{lstlisting}{}
template<class T>
void initialize (hdnum::DenseMatrix<T>& A, hdnum::Vector<T>& b)
{
  if (A.rowsize()!=A.colsize() || A.rowsize()==0)
    HDNUM_ERROR("need square and nonempty matrix");
  if (A.rowsize()!=b.size())
    HDNUM_ERROR("b must have same size as A");
  for (int i=0; i<A.rowsize(); ++i)
    {
      b[i] = 1.0;
      for (int j=0; j<A.colsize(); ++j)
        if (j<=i) A[i][j]=1.0; else A[i][j]=0.0;
    }
}
\end{lstlisting}}}
\end{frame}
En la siguiente sección se tratan los solucionadores de sistemas de ecuaciones.
Si el sistema de ecuaciones es lineal, se utiliza el método de descomposición LR 
o QR.  En el caso no lineal, como por ejemplo en el archivo num1, que utiliza 
iteraciones de punto fijo, se utiliza el método de Newton.

\subsection{Método LR}

\subsubsection{Breve explicación del algoritmo}
El método de descomposición LR se utiliza para resolver un sistema
de ecuaciones de la forma $Ax=b$. Se busca descomponer la matriz de coeficientes 
$A$ que es cuadrada en dos matrices, una matriz $L$ matriz triangular inferior y 
otra matriz $R$ matriz triangular superior, de tal forma que el producto sea
$A=LR$. Cuando es necesario realizar intercambio de filas ésto conduce a un 
sistema de la forma $PA = LR$.  Los pivotes aseguran que los elementos de la Diagonal 
no son cero, de lo contrario no se puede usar el algorítmo.

Por un lado, se hace una distinción entre pivotamiento parcial, que asegura que el
elemento más grande de la columna debajo de la diagonal en términos de cantidad se
intercambie en la diagonal mediante permutaciones de fila. Con la rotación total, uno mira la
matriz completa debajo de la diagonal y busca el elemento más grande en términos de
cantidad para cambiarlo a la entrada de la diagonal actual usando operaciones de fila y
columna. El elemento más grande en términos de valor absoluto se elige para reducir los
errores numéricos.

\subsubsection{Como hacer la descomposición LR}
Para hacer un programa que resuelva un sistema $Ax=b$ utilizando la descomposición 
LR, se procede de la siguiente forma:
\begin{itemize}
\item Se crea el vector $b$ y la matriz $A$. veamos el siguiente ejemplo:

  {\footnotesize{\begin{lstlisting}{}
Vector<number> b(3);
b[0]=15;
b[1]=73;
b[2]=12;

DenseMatrix <number> A(3,3);
A[0][0]=2;   A[0][1]=1;   A[0][2]=7;
A[1][0]=8;   A[1][1]=8;   A[1][2]=33;
A[2][0]=-4;  A[2][1]=10;  A[2][2]=4;
\end{lstlisting}}}

\item Además, necesitamos los vectores $x$ y $p$. Si se realiza un pivoteo total, 
se debe crear otro vector $q$. Para mejorar la condición de la matriz, 
las funciones \lstinline{row_equilibrate} y \lstinline{apply_equilibrate} se pueden necesitar
para lo cual se requiere un vector adicional $s$:

  {\footnotesize{\begin{lstlisting}{}
Vector<number> x(3,0.0);
Vector<number> s(3);
Vector<std::size_t> p(3);
Vector<std::size_t> q(3);
\end{lstlisting}}}

\item Como ya se mencionó en el punto anterior, el condicionamiento de la matriz $A$ 
  se puede mejorar desde el principio. Esto se hace usando las funciones \lstinline{row_equilibrate}
  y \lstinline{apply_equilibrate}. La aplicación de los comandos se puede ver en los siguientes ejemplos.

\item Ahora se aplica una de las siguientes funciones a la matriz $A$ y al vector de permutación 
$p$ creado previamente. En nuestro ejemplo realizamos un pivoteo total, por lo que 
necesitamos el vector adicional $q$:
  {\footnotesize{\begin{lstlisting}{}
row_equilibrate (A,s);
lr_fullpivot(A,p,q);
\end{lstlisting}}}
La función \lstinline{lr_partialpivot} se utilizará para el pivoteo parcial 
y la función \lstinline{lr} se utilizará para la descomposición LR sin pivotar.
(El vector de permutación adicional $q$ no es necesario aquí.) Ahora podemos 
usar el sistema de ecuaciones para resolver distintos lados derechos.

\item Para esto necesitamos preparar el lado derecho de la siguiente forma:

{\footnotesize{\begin{lstlisting}{}
apply_equilibrate(s,b);
permute_forward(p,b);
\end{lstlisting}}}

\item Luego llamamos a la función \lstinline{solveL}, que recibe la matriz $A$, 
el vector de la derecha por un vector $y$ como parámetros en los que 
se almacena la solución del sistema de ecuaciones $Ly = b$. 
Para ahorrar espacio de almacenamiento se puede 
escribir el resultado en el vector ya existente $b$.

\item Finalmente, se requiere la función \lstinline{solveR}, que resuelve el 
sistema de ecuaciones $Rx = y$. La función necesita la matriz $A$, el vector de la 
derecha y (del sistema de ecuaciones $Ly = b$) así como el vector $x$, 
en el que se guarda el resultado final:

  {\footnotesize{\begin{lstlisting}{}
solveL(A,b,b);
solveR(A,x,b);
\end{lstlisting}}}

\item Si se ha realizado un pivoteo total, las permutaciones que se almacenaron
 en el vector $q$ (transformaciones de columna de $A$) deben aplicarse al resultado
 $x$ usando \lstinline{permute_backward}:

  {\footnotesize{\begin{lstlisting}{} permute_backward(q,x);
\end{lstlisting}}}

\item La solución del sistema lineal de ecuaciones ahora se almacena en el vector x. 
En nuestro caso:
    {\footnotesize{\begin{lstlisting}{}
x[0] = 1
x[1] = 2
x[2] = 3
\end{lstlisting}}}
  \end{itemize}



\subsection{Explicaciones detalladas del algoritmo \lstinline{lr.hh}}
\begin{itemize}
\item \textbf{La función \lstinline{lr}}: Al principio se comprueba para todas 
las funciones si existe una matriz cuadrada no vacía y si el vector $p$ 
es compatible con la matriz dada. El primer bucle for busca una fila de
la matriz cuyo elemento diagonal no sea igual a cero. A continuación, 
esta línea se compara con la de la corriente Todos los elementos diagonales
intercambiados. Las permutaciones que resultan de la búsqueda pivote 
se almacenan en el vector $p$.

{\footnotesize{\begin{lstlisting}{}
for (std::size_t k=0; k<A.rowsize()-1; ++k)
{
  // finde Pivotelement und vertausche Reihen
  for (std::size_t r=k; r<A.rowsize(); ++r)
    if (A[r][k]!=0)
    {
      p[k] = r;   // speichere Permutation im Schritt k
      if (r>k)    //tausche komplette Reihe falls r!=k
        for (std::size_t j=0; j<A.colsize(); ++j)
        {
          T temp(A[k][j]);
          A[k][j] = A[r][j];
          A[r][j] = temp;
        }
      break;
    }
    if (A[k][k]==0) HDNUM_ERROR("matrix is singular");
    // Moifikation
    for (std::size_t i=k+1; i<A.rowsize(); ++i)
    {
      T qik(A[i][k]/A[k][k]);
      A[i][k] = qik;
        for (std::size_t j=k+1; j<A.colsize(); ++j)
          A[i][j] -= qik * A[k][j];
    }
}
\end{lstlisting}}}

  En el segundo ciclo $for$, se crea la matriz triangular con la matriz A
  permutada.

\item \textbf{La función \lstinline{lr_partialpivot}}: Parametros :
  Matriz $A$ y vector de permutación $p$.  Ésta función realiza un pivoteo parcial.
  Procede de la siguiente manera: Primero, el vector $p$ se inicializa describiéndolo
  con los valores de $0$ a $n-1$ (Donde $A \in \mathbb{R^{\mathrm{n}\times \mathrm{n}}}$).  
  Luego, el elemento pivote (el elemento más grande en términos de magnitud) se busca en la columna actual
  debajo del elemento diagonal en la matriz $A$ y la permutación requjerida para cambiarlo
  a la diagonal que se almacena en el vector $p$:

{\footnotesize{\begin{lstlisting}{}
for (std::size_t k=0; k<A.rowsize()-1; ++k)
{
   // finde Pivotelement
  for (std::size_t r=k+1; r<A.rowsize(); ++r)
    if (abs(A[r][k])>abs(A[k][k]))
    p[k] = r; // speichert Permutation im Schritt k
}
\end{lstlisting}}}
  En el siguiente ciclo, las líneas $k$ y $j$ se intercambian de modo que 
  el elemento pivote se encuentre en la diagonal.

\item  \textbf{La función \lstinline{lr_fullpivot}}: Funciona de manera similar a la función 
  anterior \lstinline{lr_partialpivot}, pero necesita un vector adicional $q$ para poder realizar un pivoteo total. 
  No solo es posible intercambiar filas, sino también intercambiar columnas, 
  que se almacenan en el vector $q$.

\item \textbf{La función \lstinline{permute_forward}}: 
  El vector $p$ ha almacenado las permutaciones necesarias. En esta función, las permutaciones 
  de línea se transfieren al vector:

{\footnotesize{\begin{lstlisting}{}
for (std::size_t k=0; k<b.size()-1; ++k)
  if (p[k]!=k)
  {
     T temp(b[k]);
     b[k] = b[p[k]];
     b[p[k]] = temp;
  }
\end{lstlisting}}}

\item \textbf{La función \lstinline{permute_backward}}: 
  Ésta función se utiliza al final del algorítmo LR para deshacer las permutaciones 
  realizadas en la función \lstinline{permute_forward} para el vector de la derecha.

\item \textbf{La función \lstinline{row_equilibrate}}: 
  Esta función se utiliza antes del algoritmo actual para mejorar la condición de la matriz de 
  (equilibrio). Los valores por los que se dividen las filas de la matriz se almacenan en el vector $s$:

{\footnotesize{\begin{lstlisting}{}
for (std::size_t k=0; k<A.rowsize(); ++k)
{
  s[k] = T(0.0);
    for (std::size_t j=0; j<A.colsize(); ++j)
      s[k] += abs(A[k][j]);
  if (s[k]==0) HDNUM_ERROR("row sum is zero");
    for (std::size_t j=0; j<A.colsize(); ++j)
      A[k][j] /= s[k];
}
\end{lstlisting}}}


\item \textbf{La función \lstinline{apply_equilibrate}}: 
  Los cambios que se hicieron a la matriz $A$ también se aplican aquí al vector $b$ 
  para obtener la solución correcta.

\item \textbf{La función \lstinline{solveL}}: Paramétros : Vector $x$
  y el vector  $b$.  Ésta función resuelve la ecuación $Lx=b$. Aquí $x$ se 
  determina iterativamente de la siguiente forma:
  $x_i= b_i - \sum_{j=0}^{i-1} l_{ij}x_j$

{\footnotesize{\begin{lstlisting}{}
for (std::size_t i=0; i<A.rowsize(); ++i)
{
  T rhs(b[i]);
  for (std::size_t j=0; j<i; j++)
    rhs -= A[i][j] * x[j];
  x[i] = rhs;
}
\end{lstlisting}}}

\item \textbf{La función \lstinline{solveR}}: 
  Ésta función resuelve la ecuación $Rx=b$. 
  Por lo tanto el vector $x$ se determina de la siguiente manera:
  $x_i= b_i - \sum_{j=i+1}^{n-1}r_{ij}x_j$ (aquí está
  $R \in \mathbb{R}^{n \times n}$)

{\footnotesize{\begin{lstlisting}{}
for (int i=A.rowsize()-1; i>=0; --i)
{
  T rhs(b[i]);
  for (std::size_t j=i+1; j<A.colsize(); j++)
    rhs -= A[i][j] * x[j];
  x[i] = rhs/A[i][i];
}
\end{lstlisting}}}

\end{itemize}



\subsection{Iterationsverfahren - die Datei \lstinline{newton.hh}}

Jetzt wissen wir, wie wir lineare Gleichungssysteme der Form $Ax = b$
lösen können. Was ist jedoch zu tun, wenn das Gleichungssystem nicht
linear ist, z.B. im simplen, eindimensionalen Fall $x^2 = a$? In der
Vorlesung lernt man Verfahren, die sich Fixpunktiterationen zunutze
machen, um der Lösung sehr nahe zu kommen. Die Datei
\lstinline{newton.hh} stellt hilfreiche Werkzeuge bereit, um derartige
Gleichungen und Gleichungssystem zu lösen. Das wichtigste ist das
Newtonverfahren, mit dem man nichtlineare Gleichungen der Form
$F(x) = 0$ lösen kann. Zunächst betrachten wir jedoch die konkrete
Formulierung eines Problems in einer Klasse.



\subsubsection{Die Klasse \lstinline{SquareRootProblem}}
Um ein nichtlineares Gleichungssystem der Form $f(x)=0$ lösen zu
können müssen wir zu Beginn eine Klasse für unser Problem
erstellen. Diese benötigt neben einem geeigneten Konstruktor und den
Angaben über die Dimension des Problems eine Methode, die den
Funktionswert bereitstellt, und eine andere, welche die Ableitung der
Funktion bereitstellt.  Wir zeigen dies beispielhaft an der Klasse
SquareRootProblem:

{\footnotesize{\begin{lstlisting}{}
class WurzelProblem
{
   public:
    typedef std::size_t size_type;
    typedef N number_type;
    WurzelProblem (number_type a_);
    std::size_t size () const;
    void F (const Vector<N>& x, Vector<N>& result) const;
    void F_x (const Vector<N>& x, DenseMatrix<N>& result) const;

  private:
    number_type a;
};
\end{lstlisting}}}


\begin{itemize}
\item Typedef:

  {\footnotesize{\begin{lstlisting}{}
typedef std::size_t size_type;
typedef N number_type;
\end{lstlisting}}}

Bei den Typedefs am Anfang handelt es sich zwar nicht um Methoden,
diese sind jedoch ebenso wichtig. Wir wissen von vornherein nicht,
welcher Datentyp letztlich verwendet wird. Die Typedefs sind da, damit
der Solver später erkennen kann, mit welchem Zahlentyp die Klasse
eigentlich arbeitet.

\item Konstruktor:
  {\footnotesize{\begin{lstlisting}{}
WurzelProblem::WurzelProblem (number_type a_)
: a(a_)
{}
\end{lstlisting}}}

Damit haben wir die Möglichkeit unterschiedliche Probleme der Form
$x^2=a$ zu lösen, indem wir dem Konstruktor das gewünschte a
übergeben.

\item Dimension:

  {\footnotesize{\begin{lstlisting}{}
std::size_t Wurzelproblem::size () const
{
  return 1;
}
\end{lstlisting}}}

\item Funktionswert: $f(x)=x^2-a$:

  {\footnotesize{\begin{lstlisting}{}
void Wurzelproblem::F (const Vector<N>& x, Vector<N>& result) const
{
  result[0] = x[0]*x[0] - a;
}
\end{lstlisting}}}

Wir benötigen diese spezielle Form, da wir nur Probleme der Form
$f(x)=0$ lösen können.

\item Ableitung: $f'(x)=2x$:
  {\footnotesize{\begin{lstlisting}{}
void Wurzelprobelem::F_x (const Vector<N>& x,
    DenseMatrix<N>& result) const
{
  result[0][0] = number_type(2.0)*x[0];
}
\end{lstlisting}}}
(Die Ableitung muss manuell berechnet werden.)
\end{itemize}

\noindent
Nachdem wir unsere Problemklasse erstellt haben, können wir nun ein
Objekt dieser Klasse z.B. das Problem $x^2=5$ erstellen und dieses mit
dem Newtonalgorithmus lösen. Dazu gehen wir wie folgt vor:
\begin{itemize}
  \item Objekt der Klasse \lstinline{WurzelProblem} mit dem Namen
    \glqq problem\grqq{} erstellen, welches die Gleichung $x^2=5$
    repräsentiert:

  {\footnotesize{\begin{lstlisting}{}
WurzelProblem<Number> problem(5.0);
\end{lstlisting}}}
\end{itemize}

\noindent
Nun müssen wir ein Objekt der Klasse Newton erstellen und diverse
Parameter setzen:

\subsubsection{Die Klasse \lstinline{Newton}}
\begin{itemize}
\item Folgendermaßen kann man eine Instanz der Klasse
  \lstinline{Newton} erstellen und alle Parameter setzen:

    {\footnotesize{\begin{lstlisting}{}
Newton newton;                    // Ein Newtonobjekt
newton.set_maxit(20);             // maximale Anzahl der Iteratioenen
newton.set_verbosity(2);          // Ausfuehrlichkeit der Ausgaben
newton.set_reduction(1e-100);     // Reduktionsfaktor
newton.set_abslimit(1e-100);      // maximaler absoluter Fehler
newton.set_linesearchsteps(3);    // Wie viele Schritte fuer Linesearch
\end{lstlisting}}}

\item Schließlich benötigen wir noch einen Vektor $u$, in dem die
  Lösung gespeichert wird. Dieser muss die selbe Größe wie unser
  Problem haben:

  {\footnotesize{\begin{lstlisting}{}
Vector<Number> u(problem.size());
\end{lstlisting}}}

\item Den Startwert für das Newton Verfahren setzten wir hier auf
  17. Es kann natürlich auch ein anderer Wert gewählt werden, man muss
  allerdings beachten, dass der Startwert nicht zu weit von der Lösung
  entfernt ist, da das Newton-Verfahren nicht global konvergent ist.

  {\footnotesize{\begin{lstlisting}{}
u[0]=17.0;
\end{lstlisting}}}

\item Jetzt können wir die Methode \lstinline{solve} der Klasse Newton
  auf unser Problem anwenden:

  {\footnotesize{\begin{lstlisting}{}
newton.solve(problem,u);
\end{lstlisting}}}

\item Wir bekommen als Lösung dieses speziellen Wurzelproblems das Ergebnis:\\
  $u= 2.2361e+00$
\end{itemize}

Man kann solche Probleme nicht nur mit dem Newton-Verfahren lösen, wie
bereits gesehen, sondern auch mittels der Klasse \lstinline{Banach}.

\subsubsection{Ausführliche Erläuterungen zur Klasse Newton}

Die Klasse Newton besteht im Wesentlichen aus einer Methode
\lstinline{solve} welche zum Lösen von nichtlinearen Gleichungen
benutzt werden kann. Neben dieser Methode gibt es noch einige
Verfahrens-Parameter wie die maximale Iterationsanzahl, welche im
Konstruktor gesetzt werden können.

Beim Lösen wird zuerst überprüft ob das Residuum $r=F(x)$ bereits
kleiner oder gleich der Schranke \lstinline{abslimit} ist. Ist dies
der Fall, so ist der Startwert bereits gut genug und wir sind
fertig. Im anderen Fall wird mittels der LR-Zerlegung die nächste
Suchrichtung $\bigtriangledown f(x_k)^{-1}f(x_k) = z_k$
bestimmt. Mittels einer simplen Line-Search Methode wird ein
geeignetes $\lambda$ bestimmt und $x_{k+1}=x_k-\lambda z_k$ gesetzt.

{\footnotesize{\begin{lstlisting}{}
for (size_type k=0; k<linesearchsteps; k++)
{
  y = x;
  y.update(-lambda,z);                       // y = x-lambda*z
  model.F(y,r);                              // r = F(y)
  N newR(norm(r));                           // berechnet Norm
}
if (newR<(1.0-0.25*lambda)*R)                // pruefe Konvergenz
{
  x = y;
  R = newR;
  break;
}
else lambda *= 0.5;                          // reduziert Daempgungsfaktor
if (R<=reduction*R0)                         // pruefe Konvergenz
{
  converged = true;
  return;
}
\end{lstlisting}}}


\subsubsection{Die Klasse \lstinline{Banach}}
\begin{itemize}
\item Löst ein nichtlineares System der Form $F(x)=0$ mittels
  Fixpunktiteration $x = x - \sigma*F(x)$
\item Die wichtigste Funktion ist die Funktion \lstinline{solve}, die
  die eigentliche Lösung durchführt.
\item Bei diesem Verfahren macht man sich den banachschen Fixpunktsatz
  zunutze.
\item Eine konkrete Implementierung, wie ein Problem mit der Klasse
  \lstinline{Banach} gelöst wird ist nicht in dieser Dokumentation
  enthalten, da dies sehr ahnlich zur Lösung mit der Klasse
  \lstinline{Newton} funktioniert. Ein Beispiel sieht man in der Datei
  \lstinline{wurzelbanach.cc}. Der einzige Unterschied besteht im
  Verfahrensparameter sigma, den man bei Banach noch zusätzlich
  beachten muss.
\end{itemize}




\subsubsection{Implementierung}
{\footnotesize{\begin{lstlisting}{}
class Banach
{
  typedef std::size_t size_type;
  public:
    Banach ()
      : maxit(25), linesearchsteps(10), verbosity(0),
        reduction(1e-14), abslimit(1e-30),  sigma(1.0), converged(false);
    void set_maxit (size_type n);
    void set_sigma (double sigma_);
    void set_linesearchsteps (size_type n);
    void set_verbosity (size_type n);
    void set_abslimit (double l);
    void set_reduction (double l);
    template<class M>
    void solve (const M& model, Vector<typename M::number_type> x) const;
    bool has_converged () const;

  private:
    size_type maxit;
    size_type linesearchsteps;
    size_type verbosity;
    double reduction;
    double abslimit;
    double sigma;
    mutable bool converged;
};
\end{lstlisting}}}

\begin{itemize}
\item Mit folgender Typdefinition spart man sich Schreibarbeit und es
  ist klarer, dass damit eine Größe gemeint ist.

  {\footnotesize{\begin{lstlisting}{}
typedef std::size_t size_type;
\end{lstlisting}}}

\item Im Konstruktor werden allen privaten Parametern der Klasse Werte
  zugewiesen,...

{\footnotesize{\begin{lstlisting}{}
Banach::Banach ()
  : maxit(25), linesearchsteps(10), verbosity(0),
  reduction(1e-14), abslimit(1e-30),  sigma(1.0), converged(false)
{}
\end{lstlisting}}}

\item ...die man dann mit den folgenden Funktionen nachträglich noch
  ändern kann. Der Parameter \glqq maxit\grqq{}  sorgt dafür, dass der
  später noch erläuterte Solver in keine Endlosschleife gerät, falls
  die Fixpunktiteration nicht konvergiert, sondern in diesem Fall
  abbricht und meldet, dass keine Konvergenz vorliegt.

  {\footnotesize{\begin{lstlisting}{}
void Banach::set_maxit (size_type n)
{
  maxit = n;
}
\end{lstlisting}}}

\item Hier legt man den Verfahrensparameter $\sigma$ fest.

  {\footnotesize{\begin{lstlisting}{}
void Banach::set_sigma (double sigma_)
{
  sigma = sigma_;
}
\end{lstlisting}}}

\item Wie viele Schritte soll der Solver machen, bevor er abbricht?
  Die kann man hier festlegen.

{\footnotesize{\begin{lstlisting}{}
void Banach::set_linesearchsteps (size_type n)
{
  linesearchsteps = n;
}
\end{lstlisting}}}

\item Ausgabekontrolle: Je höher die gesetzte Zahl ist, desto genauere
  Informationen zur Konvergenz werden auf der Konsole ausgegeben. Was
  die einzelnen Zahlen bedeuten sollte man sich allerdings bei Bedarf
  im Quellcode ansehen.

  {\footnotesize{\begin{lstlisting}{}
void Banach::set_verbosity (size_type n)
{
  verbosity = n;
}
\end{lstlisting}}}

\item Fehlertoleranz
  {\footnotesize{\begin{lstlisting}{}
void Banach::set_abslimit (double l)
{
  abslimit = l;
}
\end{lstlisting}}}

\item Reduktionsfaktor
  {\footnotesize{\begin{lstlisting}{}
void Banach::set_reduction (double l)
{
  reduction = l;
}
\end{lstlisting}}}

\item Mit der Methode \lstinline{solve} kann dann ein gegebenes Model
  unter Rückgriff auf die 'private Members' gelöst werden.

  {\footnotesize{\begin{lstlisting}{}
template<class M>
void Banach::solve (const M& model, Vector<typename M::number_type> x) const
{
  typedef typename M::number_type N;
  Vector<N> r(model.size());              // Residuum
  Vector<N> y(model.size());              // temporaere Loesungen

  model.F(x,r);          // berechne das nichtlineare Residuum
  N R0(norm(r));         // Norm des Anfangsresiduums
  N R(R0);               // Norm des aktuellen Residuums

  converged = false;

  // maximal so viele Iterationen wie in Matrix festgelegt sind
  for (size_type i=1; i<=maxit; i++)
  {
    if (R<=abslimit)     //pruefe Absolutbetrag des Residuums
    {
      converged = true;
      return;
    }
  \end{lstlisting}}}

Falls das vorläufige Ergebnis noch nicht genau genug war $(\leqslant
abslimit)$, geht es in die nächste Iteration, bei der zunächst der
eigentliche Iterationsschritt ausgeführt wird und anschließend mittels
Norm getestet wird, ob das Ergebnis nun genau genug ist und man
anschließend wieder zum Beginn der for-Schleife springt. Ist das
Ergebnis genau genug, hat die Funktion ihren Zweck erfüllt und wird
beendet.

{\footnotesize{\begin{lstlisting}{}
  // next iterate
  y = x;
  y.update(-sigma,r);             // y = x-sigma*z
  model.F(y,r);                   // r = F(y)
  N newR(norm(r));                // Norm berechnen

  x = y;                          // Annahme der neuen Iterierten
  R = newR;                       // Normspeicherung

  // check convergence
  if (R<=reduction*R0 || R<=abslimit)
  {
    converged = true;
    return;
  }
}
\end{lstlisting}}}

\item Der bool-Wert, den folgende Funktion zurück gibt, wird am Anfang
  immer auf false gesetzt. Löst die Funktion \lstinline{solve} das
  Gleichungssystem erfolgreich, so setzt sie den Wert auf true und als
  private Member der Klasse bleibt dieser Wert dann auch
  erhalten. Somit sagt einem diese Funktion, ob das Fixpunktverfahren
  schon mal erfolgreich war und damit auch wieder erfolgreich sein
  wird.

{\footnotesize{\begin{lstlisting}{}
bool Banach::has_converged () const
{
  return converged;
}
\end{lstlisting}}}
\end{itemize}



\section{Gewöhnliche Differentialgleichungen}

Im folgenden Kapitel soll es um das zentrale Thema der Vorlesung
Numerik 1 gehen, das Lösen von gewöhnlichen Differentialgleichungen.
Zur Wiederholung: das ist eine Gleichung bei der eine Funktion, sowie
auch Ableitungen der Funktion vorkommen und man versucht
herauszufinden, welche Funktion die Gleichung erfüllt. HDNUM stellt
einige hilfreiche Werkzeuge zum Lösen von solchen
Differentialgleichungen zur Verfügung.  Es zeigt, wie eine
Differentialgleichung aufzubereiten ist, damit sie ein Solver (wie die
im einzelnen Funktionieren sei der Vorlesung und ihren Beweisen
überlassen) lösen kann und beinhaltet zugleich mehrere solcher
Solver. Fakt ist, dass sowohl Differentialgleichungen, als auch Solver
in Klassen verpackt sind. Diese Klassen müssen bestimmte Methoden
haben, damit sie untereinander kompatible sind. Fangen wir doch einmal
mit einem Beispiel für eine Differentialgleichung an:

\subsection{Das Paradebeispiel für eine DGL in HDNUM - \lstinline{modelproblem.hh}}

\begin{itemize}
\item Diese Datei beinhaltet lediglich die Klasse
  \lstinline{ModelProblem}, welche genau die Methoden enthält, die für
  die Kompatibilität mit jedem Solver aus HDNUM nötig sind. Also muss
  jede Differentialgleiungsklasse genau diese Methodendeklarationen
  aufweißen!
\item Die komplette Information über eine Differentialgleichung ist in
  der Implementierung der Methoden enthalten.
\item Die Datei ist so geschrieben, dass Objekte der Klasse
  \lstinline{ModelProblem} Modelprobleme im Sinne der Vorlesung sind,
  kann aber für jede beliebige Differentalgleichung umgeschrieben
  werden. Dabei ist zu beachten, dass alle Funktionsköpfe der Methoden
  nicht verändert werden. Nur so bleibt die neue DGL mit unseren
  Solvern kompatibel.
\item Ein Objekt der Klasse Modelproblem entspricht dann einer zu
  lösenden Differentialgleichung.
\item Ist die Datei im Header eingebunden, kann man im Programm
  Objekte der Klasse Modelproblem erstellen und mit dem Wissen der
  nächsten Abschnitte dann auch lösen.
\end{itemize}

{\footnotesize{\begin{lstlisting}{}
template<class T, class N=T>
class ModelProblem
{
  public:
    typedef std::size_t size_type;
    typedef T time_type;
    typedef N number_type;

    ModelProblem (const N& lambda_)
      : lambda(lambda_);

    std::size_t size () const;
    void initialize (T& t0, hdnum::Vector<N>& x0) const;  //Anfangswerte
    void f (const T& t, const hdnum::Vector<N>& x,        //Funktion f
      hdnum::Vector<N>& result) const;
    void f_x (const T& t, const hdnum::Vector<N>& x,  //Jacobi Matrix von f
       hdnum::DenseMatrix<N>& result) const;

  private:
    N lambda;
};
\end{lstlisting}}}


\begin{itemize}
\item Bei den Typedefs am Anfang handelt es sich zwar nicht um
  Methoden, diese sind jedoch auch eine kurze Erklärung wert. Man
  sieht daran gut, dass es sich um eine Template-Klasse handelt und
  nie von vornherein klar ist, welcher Datentyp dann eigentlich
  verwendet wird. Die Typedefs sind da, damit der Solver später
  erkennen kann, mit welchem Zahlentyp die Modellklasse eigentlich
  arbeitet. Wir verwenden sie, damit uns die Möglichkeit bleibt, mit
  sehr genauen Datentypen (multiple precision) zu arbeiten.

\item Der Konstruktor initialisiert falls benötigt private
  Parameter. Solche muss es aber nicht immer geben.

  {\footnotesize{\begin{lstlisting}{}
template <class T, class N=T>
ModelProblem::ModelProblem (const N& lambda_)
  : lambda(lambda_)
{}
\end{lstlisting}}}

\item Mit dieser Funktion legt man fest, welche Dimension die zu
  lösende Differentialgleichung hat.

{\footnotesize{\begin{lstlisting}{}
template <class T, class N=T>
std::size_t ModelProblem::size () const
{
  return 1;
}
\end{lstlisting}}}

\item Hier legt man die Anfangswerte fest. $t_0$ ist der zeitliche
  Anfangswert, während $x_0$ der Vektor der Anfangswerte ist. Im
  eindimensionalen enthält er also nur einen Eintrag.

{\footnotesize{\begin{lstlisting}{}
template <class T, class N=T>
void ModelProblem::initialize (T& t0, hdnum::Vector<N>& x0) const
{
  t0 = 0;
  x0[0] = 1.0;
}
\end{lstlisting}}}

\item Die Funktion f beinhaltet die eigentliche
  Differentialgleichung. Dabei wird der Vektor \lstinline{result},
  also die Lösung der Funktion \lstinline{f} zum Zeitpunkt
  \lstinline{t} berechnet.

{\footnotesize{\begin{lstlisting}{}
template <class T, class N=T>
void ModelProblem::f (const T& t, const hdnum::Vector<N>& x,
  hdnum::Vector<N>& result) const
{
  result[0] = lambda*x[0];
}
\end{lstlisting}}}

\item Diese Funktion stellt die Jacobi-Matrix der Funktion
  \lstinline{f} in \lstinline{result} zur Verfügung. Diese wird von
  impliziten Solvern benötigt.

{\footnotesize{\begin{lstlisting}{}
template <class T, class N=T>
void ModelProblem::f_x (const T& t, const hdnum::Vector<N>& x,
  hdnum::DenseMatrix<N>& result) const
{
  result[0] = lambda;
}
\end{lstlisting}}}

\item Im privaten Teil der Klasse stehen eventuell benötigte Parameter.

\end{itemize}


\subsection{Anwendungsbeispiel für \lstinline{modelproblem.hh}}
Die Datei \lstinline{modelproblem_high_dim.hh} ist eine Umformulierung
der Datei \lstinline{modelproblem.hh} und stellt die
Differentialgleichung $u'(t) = \begin{pmatrix}
  5 & -2\\
  -2 & 5
\end{pmatrix}*u(t)$ mit Anfangswert
$ u(t) = \left( \begin{array}{c}1 \\\ 3 \end{array}\right)$ da. Damit
ist sie ein Beispiel für die Darstellung einer mehrdimensionalen
Differentialgleichung.

\subsection{Der Solver löst die DGL - \lstinline{modelproblem.cc}}
\begin{itemize}
\item Diese Datei ist ein Musterbeispiel zum Lösen von gewöhnlichen
  Differentialgleichungen.
\item Sie zeigt, wie man Differentialgleichungsklasse und Solverklasse
  so kombiniert, dass die Differentialgleichung gelöst und das
  Ergebnis derart in eine Datei geschrieben wird, dass man es plotten
  kann.
\end{itemize}

{\footnotesize{\begin{lstlisting}{}
#include <iostream>
#include <vector>
#include "hdnum.hh"
#include "modelproblem.hh"
#include "expliciteuler.hh"
\end{lstlisting}}}

Im Header wird neben Bibliotheken auch das Modelproblem, sowie eine
Datei zur Lösung der Differentialgleichung eingebunden. In diesem Fall
soll die Differentialgleichung mit dem expliziten Euler gelöst werden.

{\footnotesize{\begin{lstlisting}{}
int main ()
{
  typedef double Number;                // Definiert Zahlentyp
  typedef ModelProblem<Number> Model;   // Definiert Modeltyp
  Model model(-1.0);                    // Objekt der Klasse mit lambda=-1
  typedef ExplicitEuler<Model> Solver;  // Waehle einen Solver
  Solver solver(model);                 // initialisiere Solver mit Model
  solver.set_dt(0.02);                  // Setze Zeitabstaende
  hdnum::Vector<Number> times;          // Vektor fuer Zeitabstaende
  hdnum::Vector<hdnum::Vector<Number>> states; // Loesungsvektor
  times.push_back(solver.get_time());   // Anfangszeit in Vektor speichern
  states.push_back(solver.get_state()); // Anfangswert in Vektor speichern
  while (solver.get_time()<5.0-1e-6)    // Schleife zum Loesen
    {
      solver.step();
      times.push_back(solver.get_time());    // Zeit speichern
      states.push_back(solver.get_state());  // Wert speichern
    }

  gnuplot("mp2-ee-0.02.dat",times,states);   // Ausgabe wird im Abschnitt
                                             // ueber Gnuplot erklaert
  return 0;
}
\end{lstlisting}}}

Ein Alternativbeispiel ist die Datei
\lstinline{modelproblem_high_dim.cc}. Indem man beim Solver EE durch
andere Solver aus \lstinline{ode.hh} (Erklärung siehe unten) ersetzt,
kann man die DGL mit verschiedenen Mitteln lösen und sieht dabei
gleich ein Beispiel, dass eine DGL mit allen unseren Solvern
kompatibel ist.

\subsection{Was muss ein Solver können? - \lstinline{expliciteuler.hh}}
\begin{itemize}
\item Diese Datei enthält die Klasse \lstinline{ExplicitEuler}.
\item In der Klasse gibt es alle Methoden, die ein Solver in unserem
  Kontext braucht.
\item Alle Solver haben mindestens die Methoden, die
  \lstinline{ExplicitEuler} hat, eventuell noch ein paar mehr.
\item Mit Hilfe dieser Datei kann man alle Differentialgleichungen
  lösen, die die bereits erwähnte Darstellung in einer Klasse
  besitzen.
\end{itemize}

{\footnotesize{\begin{lstlisting}{}
template<class M>
class ExplicitEuler
{
  public:
    typedef typename M::size_type size_type;
    typedef typename M::time_type time_type;
    typedef typename M::number_type number_type;

    ExplicitEuler (const M& model_)
      : model(model_), u(model.size()), f(model.size());
    void set_dt (time_type dt_);
    void step ();
    const hdnum::Vector<number_type>& get_state () const;
    time_type get_time () const;
    time_type get_dt () const;

  private:  //Die private Member sind bei jedem Solver ähnlich.
    const M& model;      //Referenz auf das Model ist IMMER vorhanden
    time_type t, dt;     //Zeitlichen Variablen
    hdnum::Vector<number_type> u;  //Vektor zur Speicherung von Zeitschritten
    hdnum::Vector<number_type> f;  //mindestens einem Vektor
                  // zur Speicherung von Loesungen
};
\end{lstlisting}}}

\begin{itemize}
\item Zuerst noch eine kurze Bemerkung zu den Typedefs am Anfang: Der
  Solver hat zuanächst im Konstruktor nur eine Referenz auf ein Model
  bekommen. Damit ist aber noch nicht klar, mit welchen Zahlentypen im
  Model gearbeitet wird und ob man die Funktionen davon aufrufen
  kann. Um dies festzusetzen dienen die Typedefs. Somit kann der
  Solver DGLs für beliebige Zahlentypen lösen und erst beim
  Kompilieren wird festgelegt, welcher eigentlich gemeint ist.

\item Der Konstruktor speichert eine Referenz zu dem Model, das er
  lösen soll. Außerdem werden hier Parameter für den
  Lösungsalgorithmus wie die Größe der Zeitschritte, Anfangswerte,
  oder ähnliches festgelegt.

{\footnotesize{\begin{lstlisting}{}
template<class M>
ExplicitEuler::ExplicitEuler (const M& model_)
  : model(model_), u(model.size()), f(model.size())
{
  model.initialize(t,u);
  dt = 0.1;
}
\end{lstlisting}}}

\item Da Solver die Lösung (Zeit-)Schritt für (Zeit-)Schritt
  berechnen, kann man festlegen, wie groß diese Schritte sein
  sollen. Je größer die Schritte, desto ungenauer das Ergebnis, aber
  desto geringer der Rechenaufwand.

{\footnotesize{\begin{lstlisting}{}
template<class M>
void ExplicitEuler::set_dt (time_type dt_)
{
  dt = dt_;
}
\end{lstlisting}}}

\item Der eigentliche Lösungsalgorithmus steht in der Funktion
  step. Sie entscheidet, wie man vom einem zum nächsten Schritt
  gelangt. Hier steht also der Algorithmus des expliziten Eulers.

{\footnotesize{\begin{lstlisting}{}
template<class M>
void ExplicitEuler::step ()
{
  model.f(t,u,f);   // berechnet Wert von f an der Stelle t
  u.update(dt,f);   // naechster Funktionswert ist alter Wert+dt*f(t)
  t += dt;          // die Zeit wird um dt nach vorne gesetzt
}
\end{lstlisting}}}

\item Der bisher errechnete Lösungsvektor:
{\footnotesize{\begin{lstlisting}{}
template<class M>
const hdnum::Vector<number_type>& ExplicitEuler::get_state () const
{
  return u;
}
\end{lstlisting}}}
\item Der Zeitpunkt, der gerade berechnet wurde:
{\footnotesize{\begin{lstlisting}{}
template<class M>
time_type ExplicitEuler::get_time () const
{
  return t;
}
\end{lstlisting}}}

\item Das aktuelle dt (Schrittweite):
{\footnotesize{\begin{lstlisting}{}
template<class M>
time_type ExplicitEuler::get_dt () const
{
  return dt;
}
\end{lstlisting}}}
\end{itemize}

\subsection{Einschub: Gnuplot in \lstinline{ode.hh}}
Ein numerischer Solver kann uns natürlich keine analytische Lösung
einer DGL in Form einer konkreten Funktion liefern. Er kann uns aber
sagen, wie die Lösungsfunktion an ganz vielen Punkten aussieht. Damit
wir mit diesen vielen Zahltupeln etwas anfangen können, visualisieren
wir sie mit Gnuplot.  Folgende Template-Funktionen machen uns dies
sehr leicht und schreiben das Ergebnis im richtigen Format in eine
Datei, sodass wir es dann direkt plotten können.

\begin{enumerate}
\item
{\footnotesize{\begin{lstlisting}{}
void gnuplot (const std::string& fname, const std::vector<T> t,
 const std::vector<Vector<N> > u)
\end{lstlisting}}}

Nur für eindimensionale DGL geeignet!  Man übergibt der Funktion einen
Dateinamen (.dat) in Anführungsstrichen, sowie Zeit und
Lösungsvektor. Die Funktion sorgt dafür, dass die Daten in einer Art
Tabelle in einer Datei mit dem gewünschten Namen stehen. Diese Datei
kann man dann plotten.
\item
{\footnotesize{\begin{lstlisting}{}
void gnuplot (const std::string& fname, const std::vector<T> t,
 const std::vector<Vector<N> > u, const std::vector<T> dt)
\end{lstlisting}}}

Für zweidimensionale DGL geeignet!  Man übergibt der Funktion die
gleichen Daten wie oben und zusätzlich noch den zweiten
Lösungsvektor. Das Ergebnis ist ebenfalls analog. Man beachte beim
plotten dann allerdings die Eigenheiten der Mehrdimensionalität.
\end{enumerate}

Als Beispielvorlage kann der Code im vorhergehenden Abschnitt am Ende
gesehen werden.

Die wichtigsten Gnuplotbefehle im Terminal:
\begin{enumerate}
\item \lstinline{gnuplot} - öffnet Gnuplot
\item \lstinline{plot 'dateiname.dat' using 1:2} - plottet im zweidimensionalen unter Verwendung der Zeilen eins und zwei
\item \lstinline{plot 'dateiname.dat' using 1:2, 'dateiname.dat' using 1:3} - plottet im zweidimensionalen zwei Graphen
\item \lstinline{splot 'dateiname.dat' using 1:2:3} - plottet im dreidimensionalen
\item \lstinline{exit} - beendet gnuplot
\end{enumerate}


\subsection{Einschrittverfahren - \lstinline{ode.hh}}

Nachdem wir uns jetzt angeschaut haben, wie genau eine
Differentialgleichung und ein Solver in eine Klasse verpackt werden
müssen, damit sie untereinander kompatibel sind und wie man mit dem
Solver dann die Differentialgleichung löst, können wir dazu übergehen
uns mehrere solcher Solver anzuschauen. In der Vorlesung lernt man
dazu die impliziten und expliziten Runge-Kutta Verfahren als
wichtigste Beispiele kennen. Der explizite Euler den wir zuvor schon
als Beispiel hatten gehört auch dazu. In der Datei \lstinline{ode.hh}
sind mehrerer solche Solver implementiert. Damit man eine beliebige
Differentialgleichung (natürlich wieder in einer Klasse verpackt) mit
jedem Solver lösen kann, haben diese Solverklassen alle Methoden mit
den jeweils gleichen Funktionsköpfen. Lediglich in der Art wie diese
Funktionen dann implementiert sind unterscheiden sie sich, was dann
das einzelne Verfahren ausmacht.  Zusätzlich zu den Methoden der zuvor
behandelten Klasse \lstinline{ExplicitEuler} haben die Klassen in
\lstinline{ode.hh} noch einige zusätzliche Funktionen. Die Verfahren
mit Schrittweitensteuerung sind ebenfalls leicht abgewandelt.


\subsubsection{Die Verfahren in ode.hh}
\begin{itemize}
\item Explizite Runge-Kutta Verfahren
  \begin{itemize}
  \item\lstinline{EE} - expliziter Euler
  \item\lstinline{ModifiedEuler}
  \item\lstinline{Heun2}
  \item\lstinline{Heun3}
  \item\lstinline{Kutta3}
  \item\lstinline{RungeKutta4}
  \end{itemize}
\item Implizite Runge-Kutta Verfahren
  \begin{itemize}
  \item \lstinline{IE} - impliziter Euler
  \item \lstinline{DIRK} - Diagonal implizites Verfahren
  \end{itemize}
\item Schrittweitensteuerung
  \begin{itemize}
  \item\lstinline{RKF45}
  \item\lstinline{RE} - Richardsonextrapolation
  \end{itemize}
\end{itemize}

\subsection{Das allgemeine Runge-Kutta-Verfahren - \lstinline{RungeKutta}}
Diese Klasse ist dazu gebaut, um eine Differentialgleichung mit einem
beliebigen expliziten oder impliziten Runge-Kutta-Verfahren zu
Lösen. Die Differentialgleichung muss dabei auf die gleiche Weise wie
bisher in einer Klasse implementiert sein.

\subsubsection{Bedienung der Klasse \lstinline{RungeKutta}}
Der einzige Unterschied zur Handhabung einer anderen Solverklasse
besteht darin, dass dem Konstruktor zusätzlich noch das
Butcher-Tableau des gewünschten Verfahrens übergeben werden muss. Der
Funktionskopf im Namespace hdnum sieht folgendermaßen aus:
{\footnotesize{\begin{lstlisting}{} RungeKutta::RungeKutta (const
      M& model, DenseMatrix<number_type> A_, Vector<number_type> b_,
      Vector<number_type> c_)
\end{lstlisting}}}

Die Matrix \lstinline{A_} und die Vektoren \lstinline{b_} und
\lstinline{c_} kommen direkt aus dem Butcher Tableau. Alles Weitere
ist dann analog zu den anderen Solverklassen. N ist ein
Templateparameter der Klasse. Möchte man statt dem Newtonverfahren das
Banachverfahren zur Lösung von nichtlinearen Gleichungssystemen
verwenden, so ist Banach ein zweiter Templateparameter den man beim
Erzeugen eines Objektes davor schreibt. In diesem Fall macht es auch
Sinn dem Konstruktor als weiteres Argument am Schluss noch einen
$number\_type \ sigma\_$ zu übergeben. Der entsprechende Konstruktor
ist implementiert.  Das könnte dann folgendermaßen aussehen:
{\footnotesize{\begin{lstlisting}{} RungeKutta<Model, Banach>
      Solver(model, A, b, c, 0.5)
\end{lstlisting}}}

Dabei müsste dann model ein Modelproblem vom Typ Model sein und $A$ eine
$n \times n$ Matrix, sowie $b$ und $c$ $n$-dimensionale Vektoren. Das
Sigma im Banachverfahren wäre in diesem Fall dann 0,5.


\paragraph{Die Algorithmen hinter Funktion \lstinline{void step}}
\begin{itemize}
\item Die Funktion \lstinline{step} unterscheidet von Anfang an, ob es
  sich um ein explizites oder implizites Verfahren handelt. (Die
  Testfunktion erkennt dies am Butchertableau).
\item Im expliziten Fall sind alle Werte bekannt und in privaten
  Variablen gespeichert, um
  $u_{n}^h = u_{n-1}^h +h_n(b_1k_1+...+b_sk_s)$ mittels
  $k_1 = f(t_{n-1},u_{n-1}^h)$,
  $k_i = f(t_{n-1}+c_ih_n, u_{n-1}^h+h_n \sum_{j=1}^{i-1}a_{ij}k_j)$
  zu berechnen. Dabei wird die Funktion $f$ von der Problemklasse
  bereitgestellt

\item Im impliziten Fall gilt es $k_i = f(t_{n-1}+c_ih_n,
  u_{n-i}^h+h_n \sum_{j=1}^s a_{ij}k_j)$ für $i = 1,...,s$ zu lösen
  und damit $u_n^h = u_{n-1}^h + h_n \sum_{i = 1}^s b_ik_i $ zu
  bestimmen. Numerisch ist es jedoch einfacher, zunächst $z_i := h_n
  \sum_{j=1}^s a_{ij}k_j$ für $i = 1,...,s$ zu berechnen und dann die
  $k_i$ über $K= h_n^{-1}A^{-1}Z$ zu bestimmen. Dabei sind $K$ und $Z$
  Vektoren aus Vektoren.

  Falls $b^T$ gleich der letzten Zeile von $A$ ist, kann man sich die
  Berechnung der $k_i$ sparen und direkt $u_n^h = u_{n-1}+z_s$
  berechnen. Die nichtlinearen Gleichungssysteme bei der Berechnung
  der $z_i$ werden wahlweise mit dem Banach- oder Newtonverfahren
  gelöst, für die eine Problemklasse erstellt wird.

\item Sowohl für das Banach- als auch für das Newtonverfahren benötigt
  man eine bestimmte Problemklasse, die das zu lösende Problem
  modelliert. In unserem Fall erfüllt diesen Zweck die Klasse
  \lstinline{ImplicitRungeKuttaStepProblem}. Diese wird im Konstruktor
  mit allen wichtigen Größen der Klasse \lstinline{RungeKutta}
  initialisiert. Wichtig zu wissen ist jedoch, dass Banach- und
  Newtonverfahren keine Nullstellen von Funktionen, die Vektoren von
  Vektoren als Argument haben, berechnen können. Deshalb muss man $Z$
  als einen Vektor der Größe $n*s$ auffassen und erst danach wieder
  auf $s$ Vektoren der Größe $n$ zurückrechnen.

\item Das Herzstück der Klasse
  \lstinline{ImplicitRungeKuttaStepProblem} sind die Funktionen
  \lstinline{void F} und \lstinline{void F_x}. In der ersten wird die
  Funktion modelliert, die annuliert wird, wenn die richtigen $z_i$
  getroffen sind, während die zweite Funktion nur im Newtonverfahren
  benötigt wird und die Jacobimatrix der ersten Funktion bereitstellt.

\item Die Funktion \lstinline{F} sieht dabei folgendermaßen aus:
  \begin{align*}
    F: \mathbb{R}^{n*s} \to \mathbb{R}^{n*s}, \left( \begin{array}{c}z_1 \\...\\...\\z_s\end{array} \right) \mapsto \left( \begin{array}{c}F_1(z_1,...,z_s)\\...\\...\\F_s(z_1,...,z_s)\end{array} \right)
  \end{align*}
  wobei $F_i(z_1,...,z_s) = z_i-h_n \sum_{j=1}^s
  a_{ij}f(t_{n-1}+c_jh_n, u_{n-1}+z_j)$ für $i = 1,...,s$.

\item Die zu berechnende Jacobimatrix ist eine Blockmatrix aus
  $s\times s$ Blöcken der Größe $n \times n$. Dabei gilt für den
  $(i,j)$-ten Block:
  \begin{align}
    J_{ij} &= \frac{\partial F_i}{\partial z_j}(z_1,...,z_s) = \frac{\partial}{\partial z_j}(z_i - h_n \sum_{k=1}^s a_{ik}f(t_{n-1}+c_kh_n, u_{n-1}+z_k)) \\
           &= \delta_{ij}I-h_n \sum_{k=1}^s a_{ik} \frac{\partial}{\partial z_j}f(t_{n-1}+c_kh_n, u_{n-1}+z_k)\\
           &= \delta_{ij}I-h_n a_{ij} \frac{\partial f}{\partial z_j}(t_{n-1}+c_jh_n, u_{n-1}+z_j)
  \end{align}
  $ \frac{\partial f}{\partial z_j}$ erhalten wir dabei aus der
  Funktion \lstinline{f_x} der Differentialgleichungsklasse.
\end{itemize}

\subsubsection{Konsistenzordnungstests mit \lstinline{void ordertest}}

Mit dieser Funktion kann man die Konsistenzordnung eines allgemeinen
Runge-Kutta-Verfahrens, dessen Butchertableau man kennt,
bestimmen. Dazu ist es jedoch nötig, in der Klasse der
Differentialgleichung die exakte Lösung in der eine Funktion u
anzugeben.  Ein Beispiel dazu findet man in der Datei
\lstinline{modelproblem.hh}.  Der Funktionskopf von
\lstinline{ordertest} sieht folgendermaßen aus:
{\footnotesize{\begin{lstlisting}{} typedef double Number;
      template<class M, class S> void ordertest(const M&
      model, S solver, Number T, Number h_0, int L)
    \end{lstlisting}}}

Dabei beschreibt \lstinline{model} eine gewöhnliche
Differentialgleichung, \lstinline{solver} ist ein Löser und T der
Zeitpunkt, der für den Konsistenzordnungstest verwendet werdern
soll. $h_0$ ist die initiale Schrittweite und L die Anzahl, wie oft
$h_0$ bei der Berechnung halbiert werden soll.  Auf der Konsole wird
dann in der i-ten Zeile der Fehler im i-ten Schritt, sowie die damit
berechnete Konsistenzordnung ausgegeben.  Ein kurzes
Anwendungsbeispiel gibt es in der Datei
\lstinline{model_ordertest.cc}.

\paragraph{Berechnung der Konsistenzordnung}
\begin{itemize}
\item Für die Konsistenzordnung $\alpha$ gilt: $||u-u_h||=Ch^\alpha$
\item
  $E_{n_1,n_2}= \frac{||u(T)-u_{h_1}(T)||}{||u(T)-u_{h_2}(T)||} =
  \frac{Ch_1^\alpha}{Ch_2^\alpha}=
  \left(\frac{h_1}{h_2}\right)^\alpha$, wobei $h_i=\frac{h_0}{2^i}$
  gewählt wird.
\item
  $\alpha= \frac{log E_{n_1,n_2}}{log \left( \frac{h_1}{h_2} \right)}$
\item Im Fall, dass T nicht direkt von einem Zeitschritt getroffen
  wird, also $u_{h_i}(T)$ nicht direkt berechnet wird, muss man den
  Berechnungsalgorithmus anpassen. Dabei unterscheidet man mehrere
  Fälle. Wird T fast getroffen (Abstand kleiner als vorgegebenes
  $\epsilon$), so nimmt man diesen Wert, das heißt man vergrößert den
  letzten Schritt um maximal $\epsilon$, sodass man T genau
  trifft. Andernfalls verändert man die Schrittweite der letzten ein
  oder zwei Schritte um T genau zu treffen.
\end{itemize}



\subsection{Anwendungsbeispiele}
Im Ordner \lstinline{examples/num1} sind einige interessante Anwendungsbeispiele gegeben, bei denen man sehen kann, wie die Verfahren aus der Vorlesung in anderen Naturwissenschaften verwendet werden.

\subsubsection{Hodgkin-Huxley-Modell}
Das Hodgkin-Huxley-Modell kommt aus der Neurobiologie und beschreibt die Vorgänge an der Zellmembran einer Nevenzelle bei der Reizweiterleitung. Für genauere Erklärungen siehe \url{https://de.wikipedia.org/wiki/Hodgkin-Huxley-Modell}.

\subsubsection{n-body Problem}
Das n-body Problem ist ein Problem der Astrophysik, bei dem es um die Bewegungen von Himmelskörpern geht. Für genauere Erklärungen siehe \url{https://en.wikipedia.org/wiki/N-body_problem}.

\subsection{Van der Pol Oszillator}
Dabei handelt es sich um ein Schwingungsbeispiel, dass in unserem Fall ein gutes Beispiel für eine steife Differentialgleichung ist. Genaueres dazu gibt es unter \url{https://de.wikipedia.org/wiki/Van-der-Pol-System} bei Wikipedia.




\begin{appendices}

\section{Kleiner Programmierkurs}

\section{Unix Kommandos}
In der folgenden Tabelle sind die wichtigsten Kommandos fürs Terminal (das schwarze Fenster) zusammengestellt. Alle Worte in Großbuchstaben sind Platzhalter.


\begin{tabular}{|c|c|}\hline
   Kommando & Auswirkungen \\ \hline
   cd & gehe ins home-Verzeichnis \\ \hline
   cd ORDNERNAME & gehe in einen Ordner, dieser muss im Ordner enthalten sein, \\ & in dem man sich gerade befindet \\ \hline
   cd .. & gehe einen Ordner höher \\ \hline
   ls & zeigt an, was sich in dem Ordner befindet, \\
   & in dem man gerade ist \\ \hline
   tar cvf GEWÜNSCHTERNAME.tar & \\  Inhalt1.cc Inhalt2.cc ... Inhaltn.cc & erstellt ein Tar-Archiv \\ \hline
   tar xvf TARNAME.tar  & entpackt das Tar \\ \hline
   g++ -std=c++11 -o DATEINAME & \\  DATEINAME.cc & kompilieren (-std=c++11 braucht nicht jeder) \\ \hline
   ./DATEI & Ausführen der Datei \\ \hline


 \end{tabular}

\end{appendices}

\bibliographystyle{plain}
\bibliography{hdnum.bib}

\end{document}
