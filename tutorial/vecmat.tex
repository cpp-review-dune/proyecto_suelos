\section{Vektoren und Matrizen}

\mode<presentation>{
  \begin{frame}<presentation> \frametitle{Inhalt}
    \tableofcontents[currentsection,sectionstyle=show/hide,subsectionstyle=show/show/hide]
  \end{frame}
}

HDNUM stellt Matrix und Vektorklassen zur Verfügung.
\subsection{Vektoren}

\begin{frame}[fragile]
\frametitle{\lstinline{hdnum::Vector<T>}}
\begin{itemize}
\item \lstinline{hdnum::Vector<T>} ist ein Klassen-Template.
\item Es macht aus einem beliebigen (Zahl-)Datentypen \lstinline{T}
  einen Vektor.
\item Auch komplexe und hochgenaue Zahlen sind möglich.
\item Vektoren verhalten sich so wie man es aus der Mathematik kennt:
\begin{itemize}
\item Bestehen aus $n$ Komponenten.
\item Diese sind von $0$ bis $n-1$ (!) durchnummeriert.
\item Addition und Multiplikation mit Skalar.
\item Skalarprodukt und Norm (noch nicht implementiert).
\item Matrix-Vektor-Multiplikation
\end{itemize}
\item Die folgenden Beispiele findet man in \lstinline{vektoren.cc}
\end{itemize}
\end{frame}

\begin{frame}[fragile]
\frametitle{Konstruktion und Zugriff}
\begin{itemize}
\item Konstruktion mit und ohne Initialisierung\\
{\footnotesize{\begin{lstlisting}{}
hdnum::Vector<float> x(10);        // Vektor mit 10 Elementen
hdnum::Vector<double> y(10,3.14);  // 10 Elemente initialisiert
hdnum::Vector<float> a;            // ein leerer Vektor
\end{lstlisting}}}
\item Speziellere Vektoren\\
{\footnotesize{\begin{lstlisting}{}
hdnum::Vector<std::complex<double> >
  cx(7,std::complex<double>(1.0,3.0));
mpf_set_default_prec(1024); // Setze Genauigkeit für mpf\_class
hdnum::Vector<mpf_class> mx(7,mpf_class("4.44"));
\end{lstlisting}}}
\item Zugriff auf Element\\
{\footnotesize{\begin{lstlisting}{}
for (std::size_t i=0; i<x.size(); i=i+1)
  x[i] = i;                 // Zugriff auf Elemente
\end{lstlisting}}}
\item Vektorobjekt wird am Ende des umgebenden Blockes gelöscht.
\end{itemize}
\end{frame}

\begin{frame}[fragile]
\frametitle{Kopie und Zuweisung}
\begin{itemize}
\item Copy-Konstruktor: Erstellen eines Vektors als Kopie eines anderen
{\footnotesize{\begin{lstlisting}{}
hdnum::Vector<float> z(x); // z ist Kopie von x
\end{lstlisting}}}
\item Zuweisung nach Initialisierung, beide Vektoren
müssen die gleiche Größe haben!
{\footnotesize{\begin{lstlisting}{}
b = z;              // b kopiert die Daten aus z
a = 5.4;            // Zuweisung an alle Elemente
hdnum::Vector<double> w;   // leerer Vektor
w.resize(x.size()); // make correct size
w = x;              // copy elements
\end{lstlisting}}}
\item Ausschnitte von Vektoren\\
{\footnotesize{\begin{lstlisting}{}
hdnum::Vector<float> w(x.sub(7,3));// w ist Kopie von x[7],...,x[9]
z = x.sub(3,4);             // z ist Kopie von x[3],...,x[6]
\end{lstlisting}}}
\end{itemize}
\end{frame}

\begin{frame}[fragile]
\frametitle{Rechnen und Ausgabe}
\begin{itemize}
\item Vektorraumoperationen und Skalarprodukt\\
{\footnotesize{\begin{lstlisting}{}
w += z;            // w = w+z
w -= z;            // w = w-z
w *= 1.23;         // skalare Multiplikation
w /= 1.23;         // skalare Division
w.update(1.23,z);  // w = w + a*z
float s;
s = w*z;           // Skalarprodukt
\end{lstlisting}}}
\item Ausgabe auf die Konsole\\
{\footnotesize{\begin{lstlisting}{}
std::cout << w << std::endl;// schöne Ausgabe
w.iwidth(2);                // Stellen in Indexausgabe
w.width(20);                // Anzahl Stellen gesamt
w.precision(16);            // Anzahl Nachkommastellen
std::cout << w << std::endl;// nun mit mehr Stellen
std::cout <<cx << std::endl;// geht auch für complex
std::cout <<mx << std::endl;// geht auch für mpf\_class
\end{lstlisting}}}
\end{itemize}
\end{frame}

\begin{frame}[fragile]
\frametitle{Beispielausgabe}
{\footnotesize{\begin{lstlisting}{}
[   0]    1.204200e+01
[   1]    1.204200e+01
[   2]    1.204200e+01
[   3]    1.204200e+01

[ 0] 1.2042000770568848e+01
[ 1] 1.2042000770568848e+01
[ 2] 1.2042000770568848e+01
[ 3] 1.2042000770568848e+01
\end{lstlisting}}}
\end{frame}

\begin{frame}[fragile]
\frametitle{Hilfsfunktionen}
{\footnotesize{\begin{lstlisting}{}
zero(w);                    // das selbe wie w=0.0
fill(w,(float)1.0);         // das selbe wie w=1.0
fill(w,(float)0.0,(float)0.1); // w[0]=0, w[1]=0.1, w[2]=0.2, ...
unitvector(w,2);            // kartesischer Einheitsvektor
gnuplot("test.dat",w);      // gnuplot Ausgabe: i w[i]
gnuplot("test2.dat",w,z);   // gnuplot Ausgabe: w[i] z[i]
\end{lstlisting}}}
\end{frame}

\begin{frame}[fragile]
\frametitle{Funktionen}
\begin{itemize}
\item Beispiel: Summe aller Komponenten\\
{\footnotesize{\begin{lstlisting}{}
double sum (hdnum::Vector<double> x) {
  double s(0.0);
  for (std::size_t i=0; i<x.size(); i=i+1)
    s = s + x[i];
  return s;
}
\end{lstlisting}}}
\item Mit \textbf{Funktionentemplate}:\\
{\footnotesize{\begin{lstlisting}{}
template<class T>
T sum (hdnum::Vector<T> x) {
  T s(0.0);
  for (std::size_t i=0; i<x.size(); i=i+1)
    s = s + x[i];
  return s;
}
\end{lstlisting}}}
\item \textbf{Vorsicht}: Call-by-value erzeugt \textbf{keine} Kopie!
\end{itemize}
\end{frame}

\subsection{Matrizen}

\begin{frame}[fragile]
\frametitle{\lstinline{hdnum::DenseMatrix<T>}}
\begin{itemize}
\item \lstinline{hdnum::DenseMatrix<T>} ist ein Klassen-Template.
\item Es macht aus einem beliebigen (Zahl-)Datentypen \lstinline{T}
  eine Matrix.
\item Auch komplexe und hochgenaue Zahlen sind möglich.
\item Matrizen verhalten sich so wie man es aus der Mathematik kennt:
\begin{itemize}
\item Bestehen aus $m\times n$ Komponenten.
\item Diese sind von $0$ bis $m-1$ bzw. $n-1$ (!) durchnummeriert.
\item $m\times n$-Matrizen bilden einen Vektorraum.
\item Matrix-Vektor und Matrizenmultiplikation.
\end{itemize}
\item Die folgenden Beispiele findet man in \lstinline{matrizen.cc}
\end{itemize}
\end{frame}

\begin{frame}[fragile]
\frametitle{Konstruktion und Zugriff}
\begin{itemize}
\item Konstruktion mit und ohne Initialisierung\\
{\footnotesize{\begin{lstlisting}{}
hdnum::DenseMatrix<float> B(10,10);     // 10x10 Matrix uninitialisiert
hdnum::DenseMatrix<float> C(10,10,0.0); // 10x10 Matrix initialisiert
\end{lstlisting}}}
\item Zugriff auf Elemente\\
{\footnotesize{\begin{lstlisting}{}
for (int i=0; i<B.rowsize(); ++i)
  for (int j=0; j<B.colsize(); ++j)
    B[i][j] = 0.0;          // jetzt ist B initialisiert
\end{lstlisting}}}
\item Matrixobjekt wird am Ende des umgebenden Blockes gelöscht.
\end{itemize}
\end{frame}

\begin{frame}[fragile]
\frametitle{Kopie und Zuweisung}
\begin{itemize}
\item Copy-Konstruktor: Erstellen einer Matrix als Kopie einer anderen
{\footnotesize{\begin{lstlisting}{}
hdnum::DenseMatrix<float> D(B); // D Kopie von B
\end{lstlisting}}}
\item Zuweisung nach Initialisierung, beide Matrizen müssen gleiche Größe haben:
{\footnotesize{\begin{lstlisting}{}
hdnum::DenseMatrix<float> A(B.rowsize(),B.colsize()); // make correct size
A = B;                    // copy elements
\end{lstlisting}}}
\item Ausschnitte von Matrizen (Untermatrizen)\\
{\footnotesize{\begin{lstlisting}{}
hdnum::DenseMatrix<float> F(A.sub(1,2,3,4));// 3x4 Mat ab (1,2)
\end{lstlisting}}}
\end{itemize}
\end{frame}

\begin{frame}[fragile]
\frametitle{Rechnen mit Matrizen}
\begin{itemize}
\item Vektorraumoperationen\\
{\footnotesize{\begin{lstlisting}{}
A += B;           // A = A+B
A -= B;           // A = A-B
A *= 1.23;        // Multiplikation mit Skalar
A /= 1.23;        // Division durch Skalar
A.update(1.23,B); // A = A + s*B
\end{lstlisting}}}
\item Matrix-Vektor und Matrizenmultiplikation\\
{\footnotesize{\begin{lstlisting}{}
hdnum::Vector<float> x(10,1.0); // make two vectors
hdnum::Vector<float> y(10,2.0);
A.mv(y,x);               // y = A*x
A.umv(y,x);              // y = y + A*x
A.umv(y,(float)-1.0,x);  // y = y + s*A*x
C.mm(A,B);               // C = A*B
C.umm(A,B);              // C = C + A*B
\end{lstlisting}}}
\end{itemize}
\end{frame}

\begin{frame}[fragile]
\frametitle{Ausgabe und Hilfsfunktionen}
\begin{itemize}
\item Ausgabe von Matrizen\\
{\footnotesize{\begin{lstlisting}{}
std::cout << A.sub(0,0,3,3) << std::endl;// schöne Ausgabe
A.iwidth(2);                // Stellen in Indexausgabe
A.width(10);                // Anzahl Stellen gesamt
A.precision(4);             // Anzahl Nachkommastellen
std::cout << A << std::endl;// nun mit mehr Stellen
\end{lstlisting}}}
\item einige Hilfsfunktionen
{\footnotesize{\begin{lstlisting}{}
identity(A);
spd(A);
fill(x,(float)1,(float)1);
vandermonde(A,x);
\end{lstlisting}}}
\end{itemize}
\end{frame}

\begin{frame}[fragile]
\frametitle{Beispielausgabe}
{\footnotesize{\begin{lstlisting}{}
               0           1           2           3
  0   4.0000e+00 -1.0000e+00 -2.5000e-01 -1.1111e-01
  1  -1.0000e+00  4.0000e+00 -1.0000e+00 -2.5000e-01
  2  -2.5000e-01 -1.0000e+00  4.0000e+00 -1.0000e+00
  3  -1.1111e-01 -2.5000e-01 -1.0000e+00  4.0000e+00
\end{lstlisting}}}
\end{frame}

\begin{frame}[fragile]
\frametitle{Funktion mit Matrixargument}
Beispiel einer Funktion, die eine Matrix $A$ und einen Vektor $b$
initialisiert.

{\footnotesize{\begin{lstlisting}{}
template<class T>
void initialize (hdnum::DenseMatrix<T> A, hdnum::Vector<T> b)
{
  if (A.rowsize()!=A.colsize() || A.rowsize()==0)
    HDNUM_ERROR("need square and nonempty matrix");
  if (A.rowsize()!=b.size())
    HDNUM_ERROR("b must have same size as A");
  for (int i=0; i<A.rowsize(); ++i)
    {
      b[i] = 1.0;
      for (int j=0; j<A.colsize(); ++j)
        if (j<=i) A[i][j]=1.0; else A[i][j]=0.0;
    }
}
\end{lstlisting}}}
\end{frame}
